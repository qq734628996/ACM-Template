\section{莫队}

P1494 [国家集训队]小Z的袜子

普通莫队最终模板

询问区间随机取两个数相同的概率是多少,输出最简分数

组合数C(n+1,2)-C(n,2)=n,然后就很简单了,注意L==R的时候要特判,这时候分子分母都是0,gcd为0,除0会RE

\begin{lstlisting}
int n,m,a[MAXN];
struct query {
    int l,r,ind;
} q[MAXN];
int bsz,belong[MAXN],cnt[MAXN];
LL tot,ansa,ansb,resa[MAXN],resb[MAXN];
bool cmp(query& a, query& b) {
    if (belong[a.l]!=belong[b.l]) return belong[a.l]<belong[b.l];
    return (belong[a.l]&1)?(a.r>b.r):(a.r<b.r);
}
void prelude() {
    bsz=sqrt(n);
    FOR(i,1,n) belong[i]=i/bsz;
    sort(q+1,q+1+m,cmp);
}
void add(int x) {
    ansa+=cnt[x];
    cnt[x]++;
    ansb+=tot;
    tot++;
}
void del(int x) {
    tot--;
    ansb-=tot;
    cnt[x]--;
    ansa-=cnt[x];
}
LL gcd(LL a, LL b){ return b==0?a:gcd(b,a%b); }
void mt() {
    int l=1,r=0;
    ansa=ansb=tot=0;
    memset(cnt,0,sizeof(cnt));
    FOR(i,1,m) {
        int ql=q[i].l, qr=q[i].r, ind=q[i].ind;
        while (l<ql) del(a[l++]);
        while (l>ql) add(a[--l]);
        while (r<qr) add(a[++r]);
        while (r>qr) del(a[r--]);
        LL g=gcd(ansa,ansb);
        if (!g) {
            resa[ind]=0;
            resb[ind]=1;
            continue;
        }
        resa[ind]=ansa/g;
        resb[ind]=ansb/g;
    }
}

int main()
{

    while (scanf("%d%d", &n,&m)==2) {
        FOR(i,1,n) scanf("%d", &a[i]);
        FOR(i,1,m) {
            int l,r; scanf("%d%d", &l,&r);
            q[i]={l,r,i};
        }
        prelude();
        mt();
        FOR(i,1,m) printf("%lld/%lld\n", resa[i],resb[i]);
    }

    return 0;
}
\end{lstlisting}

https://www.cnblogs.com/WAMonster/p/10118934.html

P1903 [国家集训队]数颜色 / 维护队列

带修改询问区间不同数个数 带修莫队 模板

被卡常了只能用O2

\begin{lstlisting}
int n,q,a[MAXN],belong[MAXN],res[MAXN];
int bsz,cntq,cntm;
struct query {
    int l,r,t,ind;
} b[MAXN];
bool cmp(query& a, query& b) {
    if (belong[a.l]!=belong[b.l]) return belong[a.l]<belong[b.l];
    if (belong[a.r]!=belong[b.r]) return belong[a.r]<belong[b.r];
    return a.t<b.t;
}
struct modify {
    int pos,color;
} c[MAXN];
int ans,cnt[MAXM];
inline void add(int p) {
    if (cnt[a[p]]++==0) ans++;
}
inline void del(int p) {
    if (--cnt[a[p]]==0) ans--;
}
inline void update(int t, int ql, int qr) {
    int p=c[t].pos;
    if (ql<=p && p<=qr) {
        if (--cnt[a[p]]==0) ans--;
        if (cnt[c[t].color]++==0) ans++;
    }
    swap(a[p],c[t].color);
}

int main()
{

    scanf("%d%d", &n,&q);
    bsz=pow(n,2.0/3);
    FOR(i,1,n) {
        scanf("%d", &a[i]);
        belong[i]=i/bsz;
    }
    cntq=cntm=0;
    FOR(i,1,q) {
        char op[2]; scanf("%s", op);
        if (op[0]=='Q') {
            int l,r; scanf("%d%d", &l,&r);
            if (l>r) swap(l,r);
            cntq++;
            b[cntq]={l,r,cntm,cntq};
        } else {
            int p,x; scanf("%d%d", &p,&x);
            c[++cntm]={p,x};
        }
    }
    sort(b+1,b+1+cntq,cmp);
    int l=1, r=0, t=0;
    ans=0;
    memset(cnt,0,sizeof(cnt));
    FOR(i,1,cntq) {
        int ql=b[i].l, qr=b[i].r, qt=b[i].t, ind=b[i].ind;
        while (l<ql) del(l++);
        while (l>ql) add(--l);
        while (r<qr) add(++r);
        while (r>qr) del(r--);
        while (t<qt) update(++t,ql,qr);
        while (t>qt) update(t--,ql,qr);
        res[ind]=ans;
    }
    FOR(i,1,cntq) {
        printf("%d\n", res[i]);
    }

    return 0;
}
\end{lstlisting}

SP10707 COT2 - Count on a tree II

树上莫队 模板 树上两个结点路径不同结点权值个数

树上两结点路径转区间方法:先生成欧拉序——dfs时候进入是记录一下时间戳,出来的时候记录一下时间戳,并按时间戳顺序记录结点编号,共2*n长度的序列。对于结点u,v,如果in[u]>in[v],交换u和v;如果lca(u,v)==u,直接询问欧拉序的区间[in(u),out(v)];否则,询问区间(out(u),in(v)),并且加上一个点lca(u,v)int to[MAXN],nxt[MAXN],f[MAXN],tot;

\begin{lstlisting}
void add(int u, int v) {
    to[tot]=v;
    nxt[tot]=f[u];
    f[u]=tot++;
}
int n,m,a[MAXN];
int sz[MAXN],d[MAXN],fa[MAXN],son[MAXN],top[MAXN];
int b[MAXN],L[MAXN],R[MAXN],clk;
void prelude() {
    int cnt=0;
    unordered_map<int,int> mp;
    FOR(i,1,n) {
        if (!mp.count(a[i])) mp[a[i]]=++cnt;
    }
    FOR(i,1,n) a[i]=mp[a[i]];
}
void dfs(int u) {
    b[++clk]=u;
    L[u]=clk;
    sz[u]=1; d[u]=d[fa[u]]+1; son[u]=0;
    for (int i=f[u]; ~i; i=nxt[i]) {
        int v=to[i];
        if (v!=fa[u]) {
            fa[v]=u; dfs(v);
            sz[u]+=sz[v];
            if (sz[v]>sz[son[u]]) son[u]=v;
        }
    }
    b[++clk]=u;
    R[u]=clk;
}
void dfs(int u, int tp) {
    top[u]=tp;
    if (son[u]) dfs(son[u],tp);
    for (int i=f[u]; ~i; i=nxt[i]) {
        int v=to[i];
        if (v!=son[u] && v!=fa[u]) dfs(v,v);
    }
}
int LCA(int x, int y) {
    while (top[x]!=top[y]) d[top[x]]>=d[top[y]]?x=fa[top[x]]:y=fa[top[y]];
    return d[x]>=d[y]?y:x;
}
void init() {
    tot=d[0]=fa[1]=clk=0;
    memset(f,-1,sizeof(f));
}
int belong[MAXN],res[MAXN],cnt[MAXN],vcnt[MAXN],bsz,ans;
struct query {
    int l,r,lca,ind;
} q[MAXN];
bool cmp(query& a, query& b) {
    if (belong[a.l]!=belong[b.l]) return belong[a.l]<belong[b.l];
    return (belong[a.l]&1)?(a.r>b.r):(a.r<b.r);
}
void add(int u) {
    if (cnt[u]++==0) {
        if (vcnt[a[u]]++==0) ans++;
    } else {
        if (--vcnt[a[u]]==0) ans--;
    }
}
void del(int u) {
    if (--cnt[u]==0) {
        if (--vcnt[a[u]]==0) ans--;
    } else {
        if (vcnt[a[u]]++==0) ans++;
    }
}

int main()
{

    while (scanf("%d%d", &n,&m)==2) {
        init();
        FOR(i,1,n) scanf("%d", &a[i]);
        REP(i,n-1) {
            int u,v; scanf("%d%d", &u,&v);
            add(u,v);
            add(v,u);
        }
        prelude();
        dfs(1);
        dfs(1,1);
        FOR(i,1,m) {
            int u,v; scanf("%d%d", &u,&v);
            if (L[u]>L[v]) swap(u,v);
            int lca=LCA(u,v);
            if (lca==u) q[i]={L[u],L[v],0,i};
            else q[i]={R[u],L[v],lca,i};
        }
        bsz=sqrt(2*n);
        FOR(i,1,2*n) belong[i]=i/bsz;
        sort(q+1,q+1+m,cmp);
        int l=1,r=0;
        ans=0;
        memset(cnt,0,sizeof(cnt));
        memset(vcnt,0,sizeof(vcnt));
        FOR(i,1,m) {
            int ql=q[i].l, qr=q[i].r, lca=q[i].lca, ind=q[i].ind;
            while (l<ql) del(b[l++]);
            while (l>ql) add(b[--l]);
            while (r<qr) add(b[++r]);
            while (r>qr) del(b[r--]);
            if (lca) add(lca);
            res[ind]=ans;
            if (lca) del(lca);
        }
        FOR(i,1,m) {
            printf("%d\n", res[i]);
        }
    }

    return 0;
}
\end{lstlisting}

AT1219 歴史の研究

回滚莫队 模板

询问区间max(某数x乘上该数的出现次数)

对于添加操作容易实现,但删除操作不太可能,可以使用回滚莫队。

区别普通莫队:1.需要维护多一个块区间的最右端点br数组;2.记录上一个块的bid,每次遍历到一个新的块,把cnt数组清空(这里复杂度O(n),共需清空sqrt(n)次),把左右指针指向当前块的最右端。3.如果右指针小于当前块最右端,添加元素,保存这时的答案(后面要恢复用),枚举左边部分的元素,逐个添加(注意右端点位置),更新答案,然后逐个删除,复原答案。@int n,m,a[MAXN]

\begin{lstlisting}
vector<LL> val;
struct query {
    int l,r,ind;
} q[MAXN];
int bsz,cnt[MAXN],belong[MAXN],br[MAXN];
LL ans,res[MAXN];
void prelude() {
    val.clear();
    unordered_map<int,int> mp;
    FOR(i,1,n) {
        if (!mp.count(a[i])) {
            val.pb(a[i]);
            mp[a[i]]=SZ(val)-1;
        }
        a[i]=mp[a[i]];
    }
    bsz=sqrt(n);
    FOR(i,1,n) {
        belong[i]=i/bsz;
        br[i/bsz]=i;
    }
}
bool cmp(query& a, query& b) {
    if (belong[a.l]!=belong[b.l]) return belong[a.l]<belong[b.l];
    return a.r<b.r;
}
void add(int x) {
    cnt[x]++;
    ans=max(ans,cnt[x]*val[x]);
}
void del(int x) {
    cnt[x]--;
}
void mt() {
    sort(q+1,q+1+m,cmp);
    int pre=-1;
    ans=0;
    int l,r;
    FOR(i,1,m) {
        int ql=q[i].l, qr=q[i].r, ind=q[i].ind;
        int bid=belong[ql];
        if (bid!=pre) {
            memset(cnt,0,sizeof(cnt[0])*SZ(val));
            ans=0;
            r=br[bid];
            l=br[bid];
            pre=bid;
        }
        while (r<qr) add(a[++r]);
        LL pans=ans;
        int ed=min(l,qr);
        FOR(j,ql,ed) add(a[j]);
        res[ind]=ans;
        FOR(j,ql,ed) del(a[j]);
        ans=pans;
    }
}

int main()
{

    while (scanf("%d%d", &n,&m)==2) {
        FOR(i,1,n) scanf("%d", &a[i]);
        FOR(i,1,m) {
            int l,r; scanf("%d%d", &l,&r);
            q[i]={l,r,i};
        }
        prelude();
        mt();
        FOR(i,1,m) {
            printf("%lld\n", res[i]);
        }
    }

    return 0;
}
\end{lstlisting}