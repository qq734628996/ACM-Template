\section{可持久化平衡树}

P5055 【模板】可持久化文艺平衡树

模板

可持久化

支持:

1. 在第 p 个数后插入数 xx 。

2. 删除第 p 个数。

3. 翻转区间 [l,r][l,r]

4. 查询区间 [l,r][l,r] 中所有数的和。

以上操作基于某个历史版本

可持久化只需在原来fhq\_treap基础上添加:凡是涉及到修改全部newnode,并复制原来结点信息

注意这里Merge操作中并没有新开结点,因为split和merge总是成对出现

\begin{lstlisting}
struct fhq_treap {
    #define lt T[x].ls
    #define rt T[x].rs
    struct node {
        int rnd,sz,ls,rs,rev;
        LL val,sum;
    } T[MAXN<<7];
    int root[MAXN],tot;
    void init() {
        srand(19981111);
        tot=0;
        memset(root,0,sizeof(root));
    }
    int newnode(LL val){ T[++tot]={rand(),1,0,0,0,val,val}; return tot; }
    int cp(int x) { T[++tot]=T[x]; return tot; }
    void revone(int& x) {
        if (x) {
            x=cp(x);
            T[x].rev^=1;
            swap(lt,rt);
        }
    }
    void up(int x) {
        if (x) {
            T[x].sz=T[lt].sz+T[rt].sz+1;
            T[x].sum=T[lt].sum+T[rt].sum+T[x].val;
        }
    }
    void down(int x) {
        if (x) {
            if (T[x].rev) revone(lt), revone(rt);
            T[x].rev=0;
        }
    }
    void split(int x, int& l, int& r, int k) {
        if (!x){ l=r=0; return; }
        down(x);
        if (k<=T[lt].sz) r=cp(x), split(lt,l,T[r].ls,k), up(r);
        else l=cp(x), split(rt,T[l].rs,r,k-T[lt].sz-1), up(l);
    }
    void Merge(int& x, int l, int r) {
        if (!l||!r) x=l+r;
        else if (T[l].rnd<T[r].rnd) down(x=l), Merge(rt,rt,r), up(x);
        else down(x=r), Merge(lt,l,lt), up(x); //down(x=cp(r))
    }
    void Insert(int now, int v, int pos, LL val) {
        int x,y;
        split(root[v],x,y,pos);
        Merge(x,x,newnode(val)); Merge(root[now],x,y);
    }
    void del(int now, int v, int pos) {
        int x,y,z;
        split(root[v],x,y,pos); split(x,x,z,pos-1);
        Merge(root[now],x,y);
    }
    void Reverse(int now, int v, int l, int r) {
        int x,y,z;
        split(root[v],x,y,r); split(x,x,z,l-1);
        revone(z);
        Merge(x,x,z); Merge(root[now],x,y);
    }
    LL query_sum(int now, int v, int l, int r) {
        int x,y,z;
        split(root[v],x,y,r); split(x,x,z,l-1);
        LL ans=T[z].sum;
        Merge(x,x,z); Merge(root[now],x,y);
        return ans;
    }
} solve;

int main()
{

    int q;
    while (scanf("%d", &q)==1) {
        solve.init();
        LL ans=0;
        FOR(i,1,q) {
            int v,op; scanf("%d%d", &v,&op);
            LL x,y;
            if (op==1) {
                scanf("%lld%lld", &x,&y);
                x^=ans, y^=ans;
                solve.Insert(i,v,x,y);
            } else if (op==2) {
                scanf("%lld", &x);
                x^=ans;
                solve.del(i,v,x);
            } else if (op==3) {
                scanf("%lld%lld", &x,&y);
                x^=ans, y^=ans;
                solve.Reverse(i,v,x,y);
            } else {
                scanf("%lld%lld", &x,&y);
                x^=ans, y^=ans;
                ans=solve.query_sum(i,v,x,y);
                printf("%lld\n", ans);
            }
        }
    }

    return 0;
}
\end{lstlisting}

P3835 【模板】可持久化平衡树

模板

2.42s(O2)

支持历史版本,支持:

1. 插入x数

2. 删除x数(若有多个相同的数,因只删除一个,如果没有请忽略该操作)

3. 查询x数的排名(排名定义为比当前数小的数的个数+1。若有多个相同的数,因输出最小的排名)

4. 查询排名为x的数

5. 求x的前驱(前驱定义为小于x,且最大的数,如不存在输出-2147483647)

6. 求x的后继(后继定义为大于x,且最小的数,如不存在输出2147483647)

\begin{lstlisting}
struct fhq_treap {
    #define lt T[x].ls
    #define rt T[x].rs
    struct node {
        int rnd,sz,ls,rs,sum;
    } T[MAXN<<6];
    int root[MAXN],tot;
    void init() {
        srand(19981111);
        tot=0;
    }
    int newnode(int val){ T[++tot]={rand(),1,0,0,val}; return tot; }
    int cp(int x){ T[++tot]=T[x]; return tot; }
    void up(int x){ if (x) T[x].sz=T[lt].sz+T[rt].sz+1; }
    void split(int x, int& l, int& r, int k) {
        if (!x) { l=r=0; return; }
        if (k<=T[lt].sz) r=cp(x), split(lt,l,T[r].ls,k), up(r);
        else l=cp(x), split(rt,T[l].rs,r,k-T[lt].sz-1), up(l);
    }
    void Merge(int& x, int l, int r) {
        if (!l||!r) x=l+r;
        else if (T[l].rnd<T[r].rnd) x=l, Merge(rt,rt,r), up(x);
        else x=r, Merge(lt,l,lt), up(x);
    }
    //the max pos of x s.t. x<val
    int Rank(int x, int val) {
        if (!x) return 0;
        if (T[x].sum>=val) return Rank(lt,val);
        return T[lt].sz+1+Rank(rt,val);
    }
    void Insert(int now, int v, int val) {
        int x,y;
        split(root[v],x,y,Rank(root[v],val));
        Merge(x,x,newnode(val)); Merge(root[now],x,y);
    }
    void del(int now, int v, int val) {
        int x,y,z,rnk=Rank(root[v],val);
        if (rnk>=T[root[v]].sz) { root[now]=root[v]; return; }
        split(root[v],x,y,rnk+1), split(x,x,z,rnk);
        if (T[z].sum!=val) Merge(x,x,z);
        Merge(root[now],x,y);
    }
    int kth(int now, int v, int k) {
        int x,y,z;
        split(root[v],x,y,k); split(x,x,z,k-1);
        int ans=T[z].sum;
        Merge(x,x,z); Merge(root[now],x,y);
        return ans;
    }
    int query_rank(int now, int v, int val) {
        root[now]=root[v];
        return Rank(root[v],val)+1;
    }
    int pre(int now, int v, int val) {
        int x,y,z,rnk=Rank(root[v],val);
        if (rnk<1) {
            root[now]=root[v];
            return -2147483647;
        } else {
            split(root[v],x,y,rnk); split(x,x,z,rnk-1);
            int ans=T[z].sum;
            Merge(x,x,z); Merge(root[now],x,y);
            return ans;
        }
    }
    int succ(int now, int v, int val) {
        int x,y,z,rnk=Rank(root[v],val+1);
        if (rnk>=T[root[v]].sz) {
            root[now]=root[v];
            return 2147483647;
        } else {
            split(root[v],x,y,rnk+1), split(x,x,z,rnk);
            int ans=T[z].sum;
            Merge(x,x,z); Merge(root[now],x,y);
            return ans;
        }
    }
} solve;

int main()
{

    int n;
    while (scanf("%d", &n)==1) {
        solve.init();
        FOR(i,1,n) {
            int v,op,val; scanf("%d%d%d", &v,&op,&val);
            switch (op) {
                case 1: solve.Insert(i,v,val); break;
                case 2: solve.del(i,v,val); break;
                case 3: printf("%d\n", solve.query_rank(i,v,val)); break;
                case 4: printf("%d\n", solve.kth(i,v,val)); break;
                case 5: printf("%d\n", solve.pre(i,v,val)); break;
                case 6: printf("%d\n", solve.succ(i,v,val)); break;
            }
        }
    }

    return 0;
}
\end{lstlisting}