\section{树链剖分}


Query on a tree SPOJ - QTREE

树链剖分 边权 单点修改 区间最值

给的是边权,可以通过下放边权到深度较深的那个结点解决,树剖查询的时候,如果top结点相同,查询时不查祖先那个点,如果只有祖先一个点,直接返回

要保存每条边的两个结点,然后通过深度判断第i条边的边权是下放到哪个结点去了

\begin{lstlisting}
int n,a[MAXN],val[MAXN];
#define ls x<<1
#define rs x<<1|1
#define mid (l+r)/2
struct node {
    int ma;
} T[MAXN<<2];
void up(int x) {
    T[x].ma=max(T[ls].ma,T[rs].ma);
}
void build(int x, int l, int r) {
    T[x]={a[l]};
    if (l<r) {
        build(ls,l,mid);
        build(rs,mid+1,r);
        up(x);
    }
}
void update(int x, int l, int r, int p, int val) {
    if (l==r) T[x]={val};
    else {
        if (p<=mid) update(ls,l,mid,p,val);
        else update(rs,mid+1,r,p,val);
        up(x);
    }
}
int query_max(int x, int l, int r, int ql, int qr) {
    if (ql<=l && r<=qr) return T[x].ma;
    else {
        int ans=-INF;
        if (ql<=mid) ans=max(ans,query_max(ls,l,mid,ql,qr));
        if (mid<qr) ans=max(ans,query_max(rs,mid+1,r,ql,qr));
        return ans;
    }
}
struct edge {
    int to,nxt,w;
} e[MAXM];
int tot,f[MAXN],uu[MAXN],vv[MAXN];
void add(int u, int v, int w) {
    e[tot]={v,f[u],w}; f[u]=tot++;
}
int sz[MAXN],fa[MAXN],d[MAXN],son[MAXN],top[MAXN],clk,in[MAXN];
void dfs(int u) {
    sz[u]=1; d[u]=d[fa[u]]+1; son[u]=0;
    for (int i=f[u]; ~i; i=e[i].nxt) {
        int v=e[i].to;
        if (v!=fa[u]) {
            fa[v]=u; val[v]=e[i].w; dfs(v);
            sz[u]+=sz[v];
            if (sz[v]>sz[son[u]]) son[u]=v;
        }
    }
}
void dfs(int u, int tp) {
    a[++clk]=val[u]; in[u]=clk;
    top[u]=tp;
    if (son[u]) dfs(son[u],tp);
    for (int i=f[u]; ~i; i=e[i].nxt) {
        int v=e[i].to;
        if (v!=fa[u] && v!=son[u]) dfs(v,v);
    }
}
int tree_max(int x, int y) {
    int ans=-INF;
    while (top[x]!=top[y]) {
        if (d[top[x]]<d[top[y]]) swap(x,y);
        ans=max(ans,query_max(1,1,n,in[top[x]],in[x]));
        x=fa[top[x]];
    }
    if (x==y) return ans;
    if (in[x]>in[y]) swap(x,y);
    ans=max(ans,query_max(1,1,n,in[x]+1,in[y]));
    return ans;
}
void init() {
    tot=fa[1]=clk=0;
    memset(f,-1,sizeof(f));
    val[1]=-INF;
}

int main()
{

    int kase; scanf("%d", &kase);
    while (kase--) {
        init();
        scanf("%d", &n);
        FOR(i,1,n-1) {
            int z;
            scanf("%d%d%d", &uu[i],&vv[i],&z);
            add(uu[i],vv[i],z);
            add(vv[i],uu[i],z);
        }
        dfs(1);
        dfs(1,1);
        build(1,1,n);
        FOR(i,1,n-1) {
            if (d[uu[i]]>d[vv[i]]) swap(uu[i],vv[i]);
        }
        char op[11];
        while (scanf("%s", op), op[0]!='D') {
            int x,y; scanf("%d%d", &x,&y);
            if (op[0]=='C') update(1,1,n,in[vv[x]],y);
            else printf("%d\n", tree_max(x,y));
        }
    }

    return 0;
}
\end{lstlisting}

P3379 【模板】最近公共祖先(LCA)

\begin{lstlisting}
int to[MAXM],nxt[MAXM],f[MAXN],tot;
void init() {
    tot=0;
    memset(f,-1,sizeof(f));
}
void add(int u, int v) {
    to[tot]=v;
    nxt[tot]=f[u];
    f[u]=tot++;
}
int n,q,root;
int sz[MAXN],d[MAXN],fa[MAXN],son[MAXN],top[MAXN];
void init1() {
    d[0]=0;
    fa[root]=0; //start at 0
    memset(son,0,sizeof(son));
}
void dfs(int u) {
    sz[u]=1; d[u]=d[fa[u]]+1;
    for (int i=f[u]; ~i; i=nxt[i]) {
        int v=to[i];
        if (v==fa[u]) continue;
        fa[v]=u; dfs(v);
        sz[u]+=sz[v];
        if (!son[u]||sz[v]>sz[son[u]]) son[u]=v;
    }
}
void dfs(int u, int tp) {
    top[u]=tp;
    if (son[u]) dfs(son[u],tp);
    for (int i=f[u]; ~i; i=nxt[i]) {
        int v=to[i];
        if (v==son[u]||v==fa[u]) continue;
        dfs(v,v);
    }
}
int LCA(int x, int y) {
    while (top[x]!=top[y]) d[top[x]]>=d[top[y]]?x=fa[top[x]]:y=fa[top[y]];
    return d[x]>=d[y]?y:x;
}

int main()
{

    while (scanf("%d%d%d", &n,&q,&root)==3) {
        init();
        REP(i,n-1) {
            int x,y; scanf("%d%d", &x,&y);
            add(x,y); add(y,x);
        }
        init1();
        dfs(root);
        dfs(root,root);
        while (q--) {
            int x,y; scanf("%d%d", &x,&y);
            printf("%d\n", LCA(x,y));
        }
    }

    return 0;
}
\end{lstlisting}

P3384 【模板】树链剖分

支持以下操作:

操作1: 格式: 1 x y z 表示将树从x到y结点最短路径上所有节点的值都加上z

操作2: 格式: 2 x y 表示求树从x到y结点最短路径上所有节点的值之和

操作3: 格式: 3 x z 表示将以x为根节点的子树内所有节点值都加上z

操作4: 格式: 4 x 表示求以x为根节点的子树内所有节点值之和

树链剖分可以把树上两个结点之间的路径转化为O(logn)个区间,方法是:第二遍dfs的时候维护dfs序,

\begin{lstlisting}
int n,q,root,p;
LL a[MAXN],val[MAXN];
#define ls x<<1
#define rs x<<1|1
#define mid (l+r)/2
struct node {
    int l,r;
    LL sum,add;
    void update(LL x) {
        sum+=x*(r-l+1); sum%=p;
        add+=x; add%=p;
    }
} T[MAXN<<2];
void up(int x) {
    T[x].sum=(T[ls].sum+T[rs].sum)%p;
}
void down(int x) {
    if (T[x].add) {
        T[ls].update(T[x].add);
        T[rs].update(T[x].add);
        T[x].add=0;
    }
}
void build(int x, int l, int r) {
    T[x]={l,r,0,0};
    if (l==r) T[x].sum=a[l];
    else {
        build(ls,l,mid);
        build(rs,mid+1,r);
        up(x);
    }
}
void update(int x, int l, int r, int ql, int qr, LL val) {
    if (ql<=l && r<=qr) T[x].update(val);
    else {
        down(x);
        if (ql<=mid) update(ls,l,mid,ql,qr,val);
        if (mid<qr) update(rs,mid+1,r,ql,qr,val);
        up(x);
    }
}
LL query(int x, int l, int r, int ql, int qr) {
    if (ql<=l && r<=qr) return T[x].sum;
    else {
        LL ans=0;
        down(x);
        if (ql<=mid) ans+=query(ls,l,mid,ql,qr);
        if (mid<qr) ans+=query(rs,mid+1,r,ql,qr);
        up(x);
        return ans%p;
    }
}
struct edge {
    int to,nxt;
} e[MAXN<<1];
int tot,f[MAXN];
void add(int u, int v) {
    e[tot]={v,f[u]}; f[u]=tot++;
}
int sz[MAXN],d[MAXN],fa[MAXN],son[MAXN],top[MAXN],clk,in[MAXN];
void dfs(int u) {
    sz[u]=1; d[u]=d[fa[u]]+1; son[u]=0;
    for (int i=f[u]; ~i; i=e[i].nxt) {
        int v=e[i].to;
        if (v!=fa[u]) {
            fa[v]=u; dfs(v);
            sz[u]+=sz[v];
            if (sz[v]>sz[son[u]]) son[u]=v;
        }
    }
}
void dfs(int u, int tp) {
    a[++clk]=val[u]; in[u]=clk;
    top[u]=tp;
    if (son[u]) dfs(son[u],tp);
    for (int i=f[u]; ~i; i=e[i].nxt) {
        int v=e[i].to;
        if (v!=fa[u] && v!=son[u]) dfs(v,v);
    }
}
void tree_add(int x, int y, LL val) {
    while (top[x]!=top[y]) {
        if (d[top[x]]<d[top[y]]) swap(x,y);
        update(1,1,n,in[top[x]],in[x],val);
        x=fa[top[x]];
    }
    if (in[x]>in[y]) swap(x,y);
    update(1,1,n,in[x],in[y],val);
}
LL tree_sum(int x, int y) {
    LL ans=0;
    while (top[x]!=top[y]) {
        if (d[top[x]]<d[top[y]]) swap(x,y);
        ans+=query(1,1,n,in[top[x]],in[x]);
        x=fa[top[x]];
    }
    if (in[x]>in[y]) swap(x,y);
    ans+=query(1,1,n,in[x],in[y]);
    return ans%p;
}
void init() {
    tot=clk=fa[root]=0;
    memset(f,-1,sizeof(f));
}

int main()
{

    while (scanf("%d%d%d%d", &n,&q,&root,&p)==4) {
        init();
        FOR(i,1,n) scanf("%lld", &val[i]), val[i]%=p;
        REP(i,n-1) {
            int x,y; scanf("%d%d", &x,&y);
            add(x,y); add(y,x);
        }
        dfs(root);
        dfs(root,root);
        build(1,1,n);
        while (q--) {
            int op,x,y; LL z; scanf("%d", &op);
            switch (op) {
                case 1:
                    scanf("%d%d%lld", &x,&y,&z);
                    tree_add(x,y,z%p);
                    break;
                case 2:
                    scanf("%d%d", &x,&y);
                    printf("%lld\n", tree_sum(x,y));
                    break;
                case 3:
                    scanf("%d%lld", &x,&z);
                    update(1,1,n,in[x],in[x]+sz[x]-1,z%p);
                    break;
                case 4:
                    scanf("%d", &x);
                    printf("%lld\n", query(1,1,n,in[x],in[x]+sz[x]-1));
                    break;
            }
        }
    }

    return 0;
}
\end{lstlisting}

P2486 [SDOI2011]染色

两种操作:1.将两结点之间路径染色。2.询问两结点之间路径连续颜色段数。

首先考虑区间染色问题,需要支持区间修改,即要下推标记;需要区间合并,则线段树需要维护左端点颜色和右端点颜色以及连续颜色段数,合并时需要检查两区间中间颜色是否相同。

其次考虑树上的区间合并,树链剖分思想是两个结点往上跳,所以开两个维护答案,最后合并答案,需要考虑区间的反转,具体可以手推一下就出来了。

\begin{lstlisting}
int n,q,a[MAXN],val[MAXN];
#define ls x<<1
#define rs x<<1|1
#define mid (l+r)/2
struct node {
    int lc,rc,cnt,lazy;
    void update(int val) {
        lc=rc=lazy=val;
        cnt=1;
    }
} T[MAXN<<2];
node cal(node a, node b) {
    return {a.lc,b.rc,a.cnt+b.cnt-(a.rc==b.lc),0};
}
void up(int x) {
    T[x]=cal(T[ls],T[rs]);
}
void down(int x) {
    if (T[x].lazy) {
        T[ls].update(T[x].lazy);
        T[rs].update(T[x].lazy);
        T[x].lazy=0;
    }
}
void build(int x, int l, int r) {
    if (l==r) T[x]={a[l],a[l],1,0};
    else {
        build(ls,l,mid);
        build(rs,mid+1,r);
        up(x);
    }
}
void update(int x, int l, int r, int ql, int qr, int val) {
    if (ql<=l && r<=qr) T[x].update(val);
    else {
        down(x);
        if (ql<=mid) update(ls,l,mid,ql,qr,val);
        if (mid<qr) update(rs,mid+1,r,ql,qr,val);
        up(x);
    }
}
node query(int x, int l, int r, int ql, int qr) {
    if (ql<=l && r<=qr) return T[x];
    else {
        node ans;
        down(x);
        if (ql<=mid && mid<qr) {
            node a=query(ls,l,mid,ql,qr);
            node b=query(rs,mid+1,r,ql,qr);
            ans=cal(a,b);
        } else {
            if (ql<=mid) ans=query(ls,l,mid,ql,qr);
            else ans=query(rs,mid+1,r,ql,qr);
        }
        up(x);
        return ans;
    }
}
struct edge {
    int to,nxt;
} e[MAXM];
int f[MAXN],tot;
void add(int u, int v) {
    e[tot]={v,f[u]}; f[u]=tot++;
}
int sz[MAXN],fa[MAXN],d[MAXN],son[MAXN],top[MAXN],clk,in[MAXN];
void dfs(int u) {
    sz[u]=1; d[u]=d[fa[u]]+1; son[u]=0;
    for (int i=f[u]; ~i; i=e[i].nxt) {
        int v=e[i].to;
        if (v!=fa[u]) {
            fa[v]=u; dfs(v);
            sz[u]+=sz[v];
            if (sz[v]>sz[son[u]]) son[u]=v;
        }
    }
}
void dfs(int u, int tp) {
    a[++clk]=val[u]; in[u]=clk;
    top[u]=tp;
    if (son[u]) dfs(son[u],tp);
    for (int i=f[u]; ~i; i=e[i].nxt) {
        int v=e[i].to;
        if (v!=fa[u] && v!=son[u]) dfs(v,v);
    }
}
void tree_update(int x, int y, int val) {
    while (top[x]!=top[y]) {
        if (d[top[x]]<d[top[y]]) swap(x,y);
        update(1,1,n,in[top[x]],in[x],val);
        x=fa[top[x]];
    }
    if (in[x]>in[y]) swap(x,y);
    update(1,1,n,in[x],in[y],val);
}
int tree_query(int x, int y) {
    node a{0,0,0}, b{0,0,0};
    while (top[x]!=top[y]) {
        if (d[top[x]]<d[top[y]]) swap(x,y), swap(a,b);
        if (a.cnt) a=cal(query(1,1,n,in[top[x]],in[x]),a);
        else a=query(1,1,n,in[top[x]],in[x]);
        x=fa[top[x]];
    }
    if (in[x]>in[y]) swap(x,y), swap(a,b);
    if (b.cnt) b=cal(query(1,1,n,in[x],in[y]),b);
    else b=query(1,1,n,in[x],in[y]);
    swap(a.lc,a.rc);
    return cal(a,b).cnt;
}
void init() {
    tot=fa[1]=clk=0;
    memset(f,-1,sizeof(f));
}

int main()
{

    while (scanf("%d%d", &n,&q)==2) {
        init();
        FOR(i,1,n) scanf("%d", &val[i]);
        REP(i,n-1) {
            int x,y; scanf("%d%d", &x,&y);
            add(x,y); add(y,x);
        }
        dfs(1);
        dfs(1,1);
        build(1,1,n);
        while (q--) {
            char op[2]; int x,y,z; scanf("%s%d%d", op,&x,&y);
            if (op[0]=='Q') printf("%d\n", tree_query(x,y));
            else scanf("%d", &z), tree_update(x,y,z);
        }
    }

    return 0;
}
\end{lstlisting}