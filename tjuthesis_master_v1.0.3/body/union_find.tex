\section{并查集与最小生成树}

\begin{lstlisting}
int n, m, u[MAXM], v[MAXM], w[MAXM], r[MAXM], p[MAXN];
bool cmp(const int i, const int j) { return w[i] < w[j]; }
int Find(int x) { return p[x] == x ? x : p[x] = Find(p[x]); }
bool Union(int i, int j) { int x = Find(i), y = Find(j); if (x != y) p[x] = y; return x != y; }
int Kruskal()
{
    REP(i,m) r[i] = i;
    sort(r, r+m, cmp);
    FOR(i,1,n) p[i] = i;
    int ans = 0;
    REP(i,m) {
        int e = r[i];
        if (Union(u[e], v[e])) ans += w[e];
    }
    return ans;
}

int main()
{

    while (~scanf("%d", &n)) {
        m = 0;
        FOR(i,1,n)FOR(j,1,n) {
            scanf("%d", &w[m]);
            if (j > i) {
                u[m] = i, v[m] = j;
                m++;
            }
        }
        printf("%d\n", Kruskal());
    }

    return 0;
}
\end{lstlisting}

POJ - 1182

经典题目:食物链

带权并查集,种类并查集

https://blog.csdn.net/niushuai666/article/details/6981689

不仅维护父结点,还维护与父结点的关系relation。相应的Find和Union操作不仅要更新集合域,还要更新关系域。此题巧妙利用0,1,2三个数表示关系,并利用矢量可加性来维护关系域。

POJ数据有问题,输入只有一组。多组输入会WA到去世。要不是去了洛谷OJ上交了一发发现有一组数据有多余输出,要没看到POJ上的讨论,简直WA到怀疑人生。

\begin{lstlisting}
int n, m, p[MAXN], r[MAXN]; //0 equal, 1 eat, 2 eated
int Find(int x)
{
    if (p[x] == x) return x;
    int root = Find(p[x]);
    r[x] = (r[x]+r[p[x]])%3;
    return p[x] = root;
}
bool Union(int op, int i, int j)
{
    int x = Find(i), y = Find(j);
    if (x == y) return r[i] == (op-1+r[j])%3;
    p[x] = y;
    r[x] = (-r[i]+op+2+r[j])%3;
    return true;
}

int main()
{

    scanf("%d%d", &n, &m);
    FOR(i,1,n) p[i] = i;
    int ans = 0;
    while (m--) {
        int op, x, y; scanf("%d%d%d", &op, &x, &y);
        if (x > n || y > n) ans++;
        else if (op == 2 && x == y) ans++;
        else if (!Union(op,x,y)) ans++;
    }
    printf("%d\n", ans);

    return 0;
}
\end{lstlisting}