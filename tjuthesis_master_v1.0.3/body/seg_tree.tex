\section{线段树}

单点加法 区间查询最大子段和

\begin{lstlisting}
#define ls x<<1
#define rs x<<1|1
#define mid (l+r)/2
struct node {
    LL sum,ma,lma,rma;
    void update(LL x) {
        sum+=x;
        ma+=x;
        lma+=x;
        rma+=x;
    }
} T[MAXN<<2];
void up(int x) {
    T[x].sum=T[ls].sum+T[rs].sum;
    T[x].ma=max(T[ls].rma+T[rs].lma,max(T[ls].ma,T[rs].ma));
    T[x].lma=max(T[ls].lma,T[ls].sum+T[rs].lma);
    T[x].rma=max(T[rs].rma,T[rs].sum+T[ls].rma);
}
void build(int x, int l, int r) {
    if (l==r) T[x]={0,0,0,0};
    else {
        build(ls,l,mid);
        build(rs,mid+1,r);
        up(x);
    }
}
void modify(int x, int l, int r, int p, LL val) {
    if (l==r) T[x].update(val);
    else {
        if (p<=mid) modify(ls,l,mid,p,val);
        else modify(rs,mid+1,r,p,val);
        up(x);
    }
}
node query(int x, int l, int r, int ql, int qr) {
    if (ql<=l && r<=qr) return T[x];
    if (qr<=mid) return query(ls,l,mid,ql,qr);
    if (mid<ql) return query(rs,mid+1,r,ql,qr);
    node ll=query(ls,l,mid,ql,qr);
    node rr=query(rs,mid+1,r,ql,qr);
    return {
        ll.sum+rr.sum,
        max(ll.rma+rr.lma,max(ll.ma,rr.ma)),
        max(ll.lma,ll.sum+rr.lma),
        max(rr.rma,rr.sum+ll.rma),
    };
}
\end{lstlisting}

单点修改 区间最值 区间求和

\begin{lstlisting}
int n,a[MAXN],val[MAXN];
#define ls x<<1
#define rs x<<1|1
#define mid (l+r)/2
struct node {
    int sum,ma;
} T[MAXN<<2];
void up(int x) {
    T[x].sum=T[ls].sum+T[rs].sum;
    T[x].ma=max(T[ls].ma,T[rs].ma);
}
void build(int x, int l, int r) {
    if (l==r) T[x]={a[l],a[l]};
    else {
        build(ls,l,mid);
        build(rs,mid+1,r);
        up(x);
    }
}
void update(int x, int l, int r, int p, int val) {
    if (l==r) T[x]={val,val};
    else {
        if (p<=mid) update(ls,l,mid,p,val);
        else update(rs,mid+1,r,p,val);
        up(x);
    }
}
int query_sum(int x, int l, int r, int ql, int qr) {
    if (ql<=l && r<=qr) return T[x].sum;
    else {
        int ans=0;
        if (ql<=mid) ans+=query_sum(ls,l,mid,ql,qr);
        if (mid<qr) ans+=query_sum(rs,mid+1,r,ql,qr);
        return ans;
    }
}
int query_max(int x, int l, int r, int ql, int qr) {
    if (ql<=l && r<=qr) return T[x].ma;
    else {
        int ans=-INF;
        if (ql<=mid) ans=max(ans,query_max(ls,l,mid,ql,qr));
        if (mid<qr) ans=max(ans,query_max(rs,mid+1,r,ql,qr));
        return ans;
    }
}
\end{lstlisting}

区间修改 区间求和

\begin{lstlisting}
int n;
#define ls x<<1
#define rs x<<1|1
#define mid (l+r)/2
struct node {
    int l,r,sum,lazy;
    void update(int val) {
        sum=val*(r-l+1);
        lazy=val;
    }
} T[MAXN<<2];
void up(int x) {
    T[x].sum=T[ls].sum+T[rs].sum;
}
void down(int x) {
    if (~T[x].lazy) {
        T[ls].update(T[x].lazy);
        T[rs].update(T[x].lazy);
        T[x].lazy=-1;
    }
}
void build(int x, int l, int r) {
    T[x]={l,r,0,-1};
    if (l<r) {
        build(ls,l,mid);
        build(rs,mid+1,r);
    }
}
void update(int x, int l, int r, int ql, int qr, int val) {
    if (ql<=l && r<=qr) T[x].update(val);
    else {
        down(x);
        if (ql<=mid) update(ls,l,mid,ql,qr,val);
        if (mid<qr) update(rs,mid+1,r,ql,qr,val);
        up(x);
    }
}
int query_sum(int x, int l, int r, int ql, int qr) {
    if (ql<=l && r<=qr) return T[x].sum;
    else {
        int ans=0;
        down(x);
        if (ql<=mid) ans+=query_sum(ls,l,mid,ql,qr);
        if (mid<qr) ans+=query_sum(rs,mid+1,r,ql,qr);
        up(x);
        return ans;
    }
}
struct edge {
    int to,nxt;
} e[MAXM];
int tot,f[MAXN];
void add(int u, int v) {
    e[tot]={v,f[u]}; f[u]=tot++;
}
int sz[MAXN],fa[MAXN],d[MAXN],son[MAXN],top[MAXN],clk,in[MAXN];
void dfs(int u) {
    sz[u]=1; d[u]=d[fa[u]]+1; son[u]=0;
    for (int i=f[u]; ~i; i=e[i].nxt) {
        int v=e[i].to;
        if (v!=fa[u]) {
            fa[v]=u; dfs(v);
            sz[u]+=sz[v];
            if (sz[v]>sz[son[u]]) son[u]=v;
        }
    }
}
void dfs(int u, int tp) {
    in[u]=++clk;
    top[u]=tp;
    if (son[u]) dfs(son[u],tp);
    for (int i=f[u]; ~i; i=e[i].nxt) {
        int v=e[i].to;
        if (v!=fa[u] && v!=son[u]) dfs(v,v);
    }
}
int install(int x) {
    int ans=0;
    while (x) {
        ans+=(in[x]-in[top[x]]+1)-query_sum(1,1,n,in[top[x]],in[x]);
        update(1,1,n,in[top[x]],in[x],1);
        x=fa[top[x]];
    }
    return ans;
}
void init() {
    tot=fa[1]=clk=0;
    memset(f,-1,sizeof(f));
}

int main()
{

    while (scanf("%d", &n)==1) {
        init();
        FOR(i,2,n) {
            int x; scanf("%d", &x);
            add(x+1,i);
        }
        dfs(1);
        dfs(1,1);
        build(1,1,n);
        int q; scanf("%d", &q);
        while (q--) {
            char op[11]; int x; scanf("%s%d", op,&x); x++;
            if (op[0]=='i') printf("%d\n", install(x));
            else {
                printf("%d\n", query_sum(1,1,n,in[x],in[x]+sz[x]-1));
                update(1,1,n,in[x],in[x]+sz[x]-1,0);
            }
        }
    }

    return 0;
}
\end{lstlisting}

K - Transformation HDU - 4578

涉及区间加、乘、定制修改,区间求和、平方和、立方和,虽是裸体但很有考验

与一般线段树无异,修改和查询都是先push\_down,修改还要push\_up,区别在于懒标记优先级:最高是set,其次add和mul相等,初始值时set=add=0,mul=1。更新set时,需将add,mul设为初始值;更新mul时,需将原mul和add乘以更新值;更新add时,只需简单加上add即可。区间求平方和、立方和,把多项式展开即可得到表达式。

线段树模板 1778ms

\begin{lstlisting}
#define ls x<<1
#define rs x<<1|1
struct node {
    int l,r;
    LL sum[4],lazy[4];
    void update(int op, LL k)
    {
        LL tmp;
        switch (op) {
        case 1:
            lazy[1] = (lazy[1]+k)%MOD;
            sum[3] += 3*k*sum[2] + 3*k*k*sum[1] + k*k*k*(r-l+1);
            sum[2] += 2*k*sum[1] + k*k*(r-l+1);
            sum[1] += k*(r-l+1);
            FOR(i,1,3) sum[i] %= MOD;
            break;
        case 2:
            lazy[1] = lazy[1]*k%MOD;
            lazy[2] = lazy[2]*k%MOD;
            tmp = k;
            FOR(i,1,3) sum[i] = sum[i]*tmp%MOD, tmp *= k;
            break;
        case 3:
            lazy[1] = 0;
            lazy[2] = 1;
            lazy[3] = k;
            tmp = k;
            FOR(i,1,3) sum[i] = tmp*(r-l+1)%MOD, tmp *= k;
            break;
        }
    }
} tree[MAXN<<2];
void push_up(int x)
{
    FOR(i,1,3) tree[x].sum[i] = (tree[ls].sum[i] + tree[rs].sum[i]) % MOD;
}
void push_down(int x)
{
    ROF(i,1,3) {
        LL& t = tree[x].lazy[i];
        LL tag = i==2 ? 1 : 0;
        if (t != tag) {
            tree[ls].update(i,t);
            tree[rs].update(i,t);
            t = tag;
        }
    }
}
void build(int x, int l, int r)
{
    tree[x].l = l, tree[x].r = r;
    FOR(i,1,3) tree[x].sum[i] = tree[x].lazy[i] = 0;
    tree[x].lazy[2] = 1;
    if (l<r) {
        int m = (l+r)>>1;
        build(ls,l,m);
        build(rs,m+1,r);
    }
}
void update(int x, int op, int l, int r, int val)
{
    int L = tree[x].l, R = tree[x].r;
    if (l <= L && R <= r) {
        tree[x].update(op,val);
    } else {
        push_down(x);
        int m = (L+R)>>1;
        if (l<=m) update(ls,op,l,r,val);
        if (m<r)  update(rs,op,l,r,val);
        push_up(x);
    }
}
LL query(int x, int l, int r, int p)
{
    int L = tree[x].l, R = tree[x].r;
    if (l <= L && R <= r) {
        return tree[x].sum[p];
    } else {
        push_down(x);
        LL ans = 0;
        int m = (L+R)>>1;
        if (l<=m) ans += query(ls,l,r,p);
        if (m<r)  ans += query(rs,l,r,p);
        return ans % MOD;
    }
}

int main()
{

    int n, m, op, l, r, val;
    while (scanf("%d%d", &n, &m) == 2 && n && m) {
        build(1,1,n);
        while (m--) {
            scanf("%d%d%d%d", &op,&l,&r,&val);
            if (op != 4) update(1,op,l,r,val);
            else printf("%I64d\n", query(1,l,r,val));
        }
    }

    return 0;
}
\end{lstlisting}

P - Atlantis HDU - 1542

扫描线热身

线段树、面积并、离散化

https://blog.csdn.net/wlxsq/article/details/47254571

维护一个覆盖次数cover,一个当前区间被覆盖的长度len。扫描线算法步骤:先把纵坐标离散化。从左到右扫描x坐标,每扫到矩形的左边的“竖边”加到线段树中,扫到右边的“竖边”即从线段树中删掉,每次把当前的得到的纵坐标覆盖长度乘上相邻两点横坐标之差。

离散化采用vector,下标从0开始,所以采用左开右闭区间(因为线段树下标从1开始)表示线段,即线段树上一个下标i,表示区间(Y[i-1],Y[i]]这条线段。

正常整数区间离散都是二元组(x1,x2-1)表示一个区间,上述表示方法则不再局限整数区间,实数区间也可以离散化表示。

不是很明白无需push\_down操作,可能每次区间修改必对应两次一正一反的操作?或者查询只查询树根?如果一般线段树写法维护cover,

面积并模板

\begin{lstlisting}
struct Line {
    double x,y1,y2;
    int sign;
    bool operator<(const Line& o) const { return x < o.x; }
} line[MAXN];
vector<double> Y;
int ID(double y) { return lower_bound(ALL(Y),y)-Y.begin(); }

#define ls x<<1
#define rs x<<1|1
struct node {
    int l,r,cover;
    double len;
} tree[MAXN<<2];
void push_up(int x)
{
    int L = tree[x].l, R = tree[x].r;
    if (tree[x].cover) tree[x].len = Y[R]-Y[L-1];
    else if (L == R) tree[x].len = 0;
    else tree[x].len = tree[ls].len + tree[rs].len;
}
void build(int x, int l, int r)
{
    tree[x].l = l, tree[x].r = r;
    tree[x].cover = tree[x].len = 0;
    if (l<r) {
        int m = (l+r)>>1;
        build(ls,l,m);
        build(rs,m+1,r);
    }
}
void update(int x, int l, int r, int val)
{
    int L = tree[x].l, R = tree[x].r;
    if (l <= L && R <= r) {
        tree[x].cover += val;
    } else {
        int m = (L+R)>>1;
        if (l<=m) update(ls,l,r,val);
        if (m<r)  update(rs,l,r,val);
    }
    push_up(x);
}

int main()
{

    int n, kase = 0;
    double x1,x2,y1,y2;
    while (~scanf("%d", &n) && n) {
        Y.clear();
        REP(i,n) {
            scanf("%lf%lf%lf%lf", &x1,&y1,&x2,&y2);
            line[i<<1] = (Line){x1,y1,y2,1};
            line[i<<1|1] = (Line){x2,y1,y2,-1};
            Y.pb(y1), Y.pb(y2);
        }
        n <<= 1;
        sort(line,line+n);
        sort(ALL(Y)); Y.erase(unique(ALL(Y)),Y.end());
        build(1,1,SZ(Y));
        double ans = 0;
        REP(i,n-1) {
            int l = ID(line[i].y1)+1, r = ID(line[i].y2);
            update(1,l,r,line[i].sign);
            ans += tree[1].len*(line[i+1].x-line[i].x);
        }
        printf("Test case #%d\nTotal explored area: %.2f\n\n", ++kase, ans);
    }

    return 0;
}
\end{lstlisting}

O - 覆盖的面积 HDU - 1255

线段树、面积交、离散化

https://www.cnblogs.com/lxjshuju/p/7040186.html

https://blog.csdn.net/zearot/article/details/47762543

与面积并的区别在于push\_up,线段树多维护里一个len表示覆盖两次以上的长度,而push\_up的更新有所变化

\begin{lstlisting}
struct Line {
    double x,y1,y2;
    int sign;
    bool operator<(const Line& o) const { return x<o.x; }
} line[MAXN];
vector<double> Y;
int ID(double y) { return lower_bound(ALL(Y),y)-Y.begin(); }

#define ls x<<1
#define rs x<<1|1
struct node {
    int l,r,cover;
    double len[3];
} tree[MAXN<<2];
void push_up(int x)
{
    FOR(i,1,2) {
        int j = i-tree[x].cover;
        if (j <= 0) tree[x].len[i] = tree[x].len[0];
        else if (tree[x].l == tree[x].r) tree[x].len[i] = 0;
        else tree[x].len[i] = tree[ls].len[j]+tree[rs].len[j];
    }
}
void build(int x, int l, int r)
{
    tree[x] = (node){l,r,0,Y[r]-Y[l-1],0,0};
    if (l<r) {
        int m = (l+r)>>1;
        build(ls,l,m);
        build(rs,m+1,r);
    }
}
void update(int x, int l, int r, int val)
{
    int L = tree[x].l, R = tree[x].r;
    if (l <= L && R <= r) {
        tree[x].cover += val;
    } else {
        int m = (L+R)>>1;
        if (l<=m) update(ls,l,r,val);
        if (m<r)  update(rs,l,r,val);
    }
    push_up(x);
}

int main()
{

    double x1,x2,y1,y2;
    int T; scanf("%d", &T);
    while (T--) {
        int n; scanf("%d", &n);
        Y.clear();
        REP(i,n) {
            scanf("%lf%lf%lf%lf", &x1,&y1,&x2,&y2);
            line[i<<1] = (Line){x1,y1,y2,1};
            line[i<<1|1] = (Line){x2,y1,y2,-1};
            Y.pb(y1), Y.pb(y2);
        }
        n <<= 1;
        sort(line,line+n);
        sort(ALL(Y)); Y.erase(unique(ALL(Y)),Y.end());
        double ans = 0;
        build(1,1,SZ(Y)-1);
        REP(i,n-1) {
            int l = ID(line[i].y1)+1, r = ID(line[i].y2);
            update(1,l,r,line[i].sign);
            ans += tree[1].len[2]*(line[i+1].x-line[i].x);
        }
        printf("%.2f\n", ans);
    }

    return 0;
}
\end{lstlisting}

N - Picture POJ - 1177

求覆盖面积周长

https://www.cnblogs.com/AC-King/p/7789013.html

相比求面积交,多维护:lines表示区间有多少个线段,每个线段贡献两个端点,lf、rf表示区间端点是否被覆盖,这样写push\_up区间合并时:(详见代码。。)

矩形并周长模板(覆盖面周长模板)

\begin{lstlisting}
struct Line {
    int x,y1,y2,sign;
    bool operator<(const Line& o) const { return x<o.x; }
} line[MAXN];
vector<int> Y;
int ID(int y) { return lower_bound(ALL(Y),y)-Y.begin(); }

#define ls x<<1
#define rs x<<1|1
struct node {
    int l,r,cover,lines,len[2],lf,rf;
} tree[MAXN<<2];
void push_up(int x)
{
    if (tree[x].cover) {
        tree[x].len[1] = tree[x].len[0];
        tree[x].lf = tree[x].rf = tree[x].lines = 1;
    } else {
        tree[x].lines = tree[ls].lines + tree[rs].lines - (tree[ls].rf && tree[rs].lf);
        tree[x].len[1] = tree[ls].len[1] + tree[rs].len[1];
        tree[x].lf = tree[ls].lf, tree[x].rf = tree[rs].rf;
    }
}
void build(int x, int l, int r)
{
    tree[x] = (node){l,r,0,0,Y[r]-Y[l-1],0,0,0};
    if (l<r) {
        int m = (l+r)>>1;
        build(ls,l,m);
        build(rs,m+1,r);
    }
}
void update(int x, int l, int r, int val)
{
    int L = tree[x].l, R = tree[x].r;
    if (l <= L && R <= r) {
        tree[x].cover += val;
    } else {
        int m = (L+R)>>1;
        if (l<=m) update(ls,l,r,val);
        if (m<r)  update(rs,l,r,val);
    }
    push_up(x);
}

int main()
{

    int n,x1,x2,y1,y2;
    while (~scanf("%d", &n)) {
        Y.clear();
        REP(i,n) {
            scanf("%d%d%d%d", &x1,&y1,&x2,&y2);
            line[i<<1] = (Line){x1,y1,y2,1};
            line[i<<1|1] = (Line){x2,y1,y2,-1};
            Y.pb(y1), Y.pb(y2);
        }
        n <<= 1;
        sort(line, line+n);
        sort(ALL(Y)); Y.erase(unique(ALL(Y)),Y.end());
        LL ans = 0;
        build(1,1,SZ(Y)-1);
        REP(i,n-1) {
            int l = ID(line[i].y1)+1, r = ID(line[i].y2);
            int preL = tree[1].len[1];
            update(1,l,r,line[i].sign);
            ans += abs(tree[1].len[1]-preL) + tree[1].lines*2*(line[i+1].x-line[i].x);
        }
        ans += tree[1].len[1];
        printf("%I64d\n", ans);
    }

    return 0;
}
\end{lstlisting}