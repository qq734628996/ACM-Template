\section{数位DP}



\begin{lstlisting}
typedef long long ll;
int a[20];
ll dp[20][state];//@不同题目状态不同@
ll dfs(int pos,/*state@变量@*/,bool lead/*@前导零@*/,bool limit/*@数位上界变量@*/)//@不是每个题都要判断前导零@
{
    //@递归边界,既然是按位枚举,最低位是0,那么pos==-1说明这个数我枚举完了@
    if(pos==-1) return 1;/*@这里一般返回1,表示你枚举的这个数是合法的,那么这里就需要你在枚举时必须每一位都要满足题目条件,也就是说当前枚举到pos位,一定要保证前面已经枚举的数位是合法的。不过具体题目不同或者写法不同的话不一定要返回1 @*/
    //@第二个就是记忆化(在此前可能不同题目还能有一些剪枝)@
    if(!limit && !lead && dp[pos][state]!=-1) return dp[pos][state];
    /*@常规写法都是在没有限制的条件记忆化,这里与下面记录状态是对应,具体为什么是有条件的记忆化后面会讲@*/
    int up=limit?a[pos]:9;//@根据limit判断枚举的上界up;这个的例子前面用213讲过了@
    ll ans=0;
    //@开始计数@
    for(int i=0;i<=up;i++)//@枚举,然后把不同情况的个数加到ans就可以了@
    {
        if() ...
        else if()...
        ans+=dfs(pos-1,/*@状态转移@*/,lead && i==0,limit && i==a[pos]) //@最后两个变量传参都是这样写的@
        /*@这里还算比较灵活,不过做几个题就觉得这里也是套路了@
        @大概就是说,我当前数位枚举的数是i,然后根据题目的约束条件分类讨论@
        @去计算不同情况下的个数,还有要根据state变量来保证i的合法性,比如题目@
        @要求数位上不能有62连续出现,那么就是state就是要保存前一位pre,然后分类,@
        @前一位如果是6那么这意味就不能是2,这里一定要保存枚举的这个数是合法@*/
    }
    //@计算完,记录状态@
    if(!limit && !lead) dp[pos][state]=ans;
    /*@这里对应上面的记忆化,在一定条件下时记录,保证一致性,当然如果约束条件不需要考虑lead,这里就是lead就完全不用考虑了@*/
    return ans;
}
ll solve(ll x)
{
    int pos=0;
    while(x)//@把数位都分解出来@
    {
        a[pos++]=x%10;//@个人老是喜欢编号为[0,pos),看不惯的就按自己习惯来,反正注意数位边界就行@
        x/=10;
    }
    return dfs(pos-1/*@从最高位开始枚举@*/,/*@一系列状态 @*/,true,true);//@刚开始最高位都是有限制并且有前导零的,显然比最高位还要高的一位视为0嘛@
}
int main()
{
    ll le,ri;
    while(~scanf("%lld%lld",&le,&ri))
    {
        //@初始化dp数组为-1,这里还有更加优美的优化,后面讲@
        printf("%lld\n",solve(ri)-solve(le-1));
    }
}
\end{lstlisting}

不要62 HDU - 2089

经典

统计[L,R]内不含4和62的个数。

优化一:

memset放外面,因为约束条件是62和4,是确定的,跟数本身没关系。

\begin{lstlisting}
int L,R,a[10],d[10][2];
int dp(int pos, bool if6, bool limit)
{
    if (pos==-1) return 1;
    if (!limit && ~d[pos][if6]) return d[pos][if6];
    int ans=0;
    int n=limit?a[pos]:9;
    FOR(i,0,n) {
        if (i==4) continue;
        if (if6 && i==2) continue;
        ans+=dp(pos-1,i==6,limit&&i==a[pos]);
    }
    if (!limit) d[pos][if6]=ans;
    return ans;
}
int solve(int x)
{
    int cnt=0;
    while (x) {
        a[cnt++]=x%10;
        x/=10;
    }
    return dp(cnt-1,false,true);
}

int main()
{

    memset(d,-1,sizeof(d));
    while (scanf("%d%d", &L,&R)==2 && L&&R) {
        printf("%d\n", solve(R)-solve(L-1));
    }

    return 0;
}
\end{lstlisting}

F(x) HDU - 4734

定义了f(x)=d[n]*2\^(n-1)+d[n-1]*2\^(n-2)+...+d[2]*2+d[1]*1,其中d[k]表示正整数x的第k个数位。求[0,b]中满足f(i)<f(a)的个数。

可以想到用d[pos][sum]表示状态,sum为当前的和,如果每次累加的话,需要每次清空d数组,会超时。

如果用d[pos][sum]表示pos位数不超过sum的数个数,每次累减,则只需要清空一次d数组即可。

\begin{lstlisting}
int a,b,c[11],d[11][MAXN],ma;
int dp(int pos, int sum, bool limit)
{
    if (sum<0) return 0;
    if (pos==-1) return 1;
    if (!limit && ~d[pos][sum]) return d[pos][sum];
    int ans=0;
    int up=limit?c[pos]:9;
    FOR(i,0,up) {
        ans+=dp(pos-1,sum-i*(1<<pos),limit&&i==up);
    }
    if (!limit) d[pos][sum]=ans;
    return ans;
}

int main()
{

    memset(d,-1,sizeof(d));
    int T; scanf("%d", &T);
    FOR(i,1,T) {
        scanf("%d%d", &a,&b);
        ma=0;
        for (int j=0; a; j++) {
            ma+=(a%10)*(1<<j);
            a/=10;
        }
        int cnt=0;
        while (b) {
            c[cnt++]=b%10;
            b/=10;
        }
        printf("Case #%d: %d\n", i,dp(cnt-1,ma,true));
    }

    return 0;
}
\end{lstlisting}

Round Numbers POJ - 3252

模板

求二进制中0个数不少于1个数的数个数。类似的变形还有计数01个数相等的题目

如果0减一,如果1加一,最后判断sum不大于初始值即可。需要维护limit还需要维护一个前导零lead,以便得知什么时候能开始对sum加减。

\begin{lstlisting}
int a[33],d[33][66];
int dp(int pos, int sum, bool limit, bool lead)
{
    if (pos==-1) return sum<=33;
    if (!limit && !lead && ~d[pos][sum]) return d[pos][sum];
    int ans=0, up=limit?a[pos]:1;
    FOR(i,0,up) {
        int add=i?1:-1;
        if (lead && !i) add=0;
        ans+=dp(pos-1,sum+add,limit&&i==up,lead&&!i);
    }
    if (!limit && !lead) d[pos][sum]=ans;
    return ans;
}
int solve(int x)
{
    int cnt=0;
    while (x) {
        a[cnt++]=(x&1);
        x>>=1;
    }
    return dp(cnt-1,33,true,true);
}

int main()
{

    memset(d,-1,sizeof(d));
    int L,R;
    while (scanf("%d%d", &L,&R)==2 && L&&R) {
        printf("%d\n", solve(R)-solve(L-1));
    }

    return 0;
}
\end{lstlisting}

\section{斜率DP}

$d_i=\min \left ( -a_i x_j + y_j \right ) + w_i$

$d_i=\max \left ( -a_i x_j + y_j \right ) + w_i$

令$b=-a_i x_j + y_j$

移项:$y_i=a_i x_j + b$

如果是求最小值,即求b最小值,问题即是在凸包的点中找一个点使得b最小

模板:

\begin{lstlisting}
struct point {
    LL x,y;
    point operator-(const point& p) const {
        return (point){x-p.x,y-p.y};
    }
};
inline LL cross(const point& u, const point& v)
{
    return u.x*v.y-v.x*u.y;
}
struct dequeue {
    point q[MAXN];
    int st,ed;
    void init() { st=1; ed=0; }
    void push(const point& u) {
        while (st<ed && cross(q[ed]-q[ed-1],u-q[ed-1])<=0) ed--;
        q[++ed]=u;
    }
    point pop(const LL& k) {
        while (st<ed && -k*q[st].x+q[st].y >= -k*q[st+1].x+q[st+1].y) st++;
        return q[st];
    }
} Q;
// d[i] = -k*x[j] + y[j] + w
void solve()
{
    Q.init();
    //Q.push((point){0,0});
    FOR(i,1,n) {
        point t=Q.pop(k);
        d[i]=-k*p.x+t.y+w;
        Q.push((point){x[i],y[i]});
    }
}
\end{lstlisting}

Pearls HDU - 1300

n个珍珠要买a[i]个,价格为p[i],单买要加上基础价10*p[i],也可以和贵的珍珠一起买不需要付基础价

要么单买,要么连续地买,因为比如a,b,c三种,如果a要跳过b用c的价格买,那还不如用b的价格买。

d[i]=min(d[j]+(a[i]-a[j]+10)*p[i]), (j<i)

可以用斜率优化,虽然规模其实很小。

\begin{lstlisting}
int n,a[MAXN],d[MAXN];

int main()
{

    int T; scanf("%d", &T);
    while (T--) {
        scanf("%d", &n);
        FOR(i,1,n) {
            int p;
            scanf("%d%d", &a[i],&p);
            a[i]+=a[i-1];
            d[i]=INF;
            REP(j,i) d[i]=min(d[i],d[j]+(a[i]-a[j]+10)*p);
        }
        printf("%d\n", d[n]);
    }

    return 0;
}
\end{lstlisting}

MAX Average Problem HDU - 2993

经典,最大平均值问题

求区间长度至少为k的最大的区间平均值

NOI2004年周源的论文《浅谈数形结合思想在信息学竞赛中的应用》

参考 白书 第一章 Average, Seoul, UVALive - 4726, 或参考紫书 例题8-9

这道题很坑爹就是规模太大了,并且用int会WA

用了能读到文件末尾的输入挂,对代码常数进行了很细致的优化(详细对比汝佳源码和这份代码),需要用LL否则会WA

\begin{lstlisting}
namespace IN {
    const int N=2e6+10;
    char str[N],*S=str,*T=str;
    inline char rdc()
    {
        if (S==T) {
            T=(S=str)+fread(str,1,N,stdin);
            if (S==T) exit(0);
        }
        return *S++;
    }
    inline int rd()
    {
        int x=0;
        char t=rdc();
        while (t>'9'||t<'0') t=rdc();
        while ('0'<=t&&t<='9') x=x*10+t-'0', t=rdc();
        return x;
    }
}
int n,L,a[MAXN];
int q[MAXN];
LL cmp(int x1, int x2, int x3, int x4)
{
    return (LL)(a[x2]-a[x1])*(x4-x3) - (LL)(a[x4]-a[x3])*(x2-x1);
}
bool cmp2(int x1, int x3, int x4)
{
    return (LL)(a[x4]-a[x1])*(x4-x3) >= (LL)(a[x4]-a[x3])*(x4-x1);
}

int main()
{

    while (1) {
        n=IN::rd(); L=IN::rd();
        FOR(i,1,n) a[i]=IN::rd(), a[i]+=a[i-1];
        int st=1, ed=0, ansL=1, ansR=L;
        FOR(i,L,n) {
            while (st<ed && cmp2(q[ed-1]-1,q[ed]-1,i-L)) ed--;
            q[++ed]=i-L+1;
            while (st<ed && cmp2(q[st+1]-1,q[st]-1,i)) st++;
            if (cmp(q[st]-1,i,ansL-1,ansR)>0) {
                ansL=q[st];
                ansR=i;
            }
        }
        printf("%.2f\n", (double)(a[ansR]-a[ansL-1])/(ansR-ansL+1));
    }

    return 0;
}
\end{lstlisting}

Picnic Cows HDU - 3045

给一些数,把他们分组,每组不少于T个数,每个组每个数分别减去这个组最小的数之后求和为这个组的费用,所有组费用加起来为总费用,求总费用最小。

先把数排序,斜率dp的套路就显现出来了——每个数要么单独选,要么与前面连续几个数一起选

d[i]=min(d[j]+S[i]-S[j]-(i-j)*a[j+1])

=min(-i*a[j+1]+j*a[j+1]+d[j]-S[j])+S[i]

套模板即可,初始化条件是(a[1],0),另外由于要求每个组数不少于L个,所以先把d预处理为-1,当d[i-L+1]不是-1时才进队

\begin{lstlisting}
struct point {
    LL x,y;
    point operator-(const point& p) const {
        return (point){x-p.x, y-p.y};
    }
};
inline LL cross(const point& u, const point& v) {
    return u.x*v.y - v.x*u.y;
}
struct dequeue {
    point q[MAXN];
    int st,ed;
    void init() { st=1, ed=0; }
    void push(const point& u) {
        while (st<ed && cross(q[ed]-q[ed-1],u-q[ed-1])<=0) ed--;
        q[++ed]=u;
    }
    point pop(const LL& k) {
        while (st<ed && -k*q[st].x+q[st].y >= -k*q[st+1].x+q[st+1].y) st++;
        return q[st];
    }
} Q;
int n,L;
LL a[MAXN],S[MAXN],d[MAXN];
void solve()
{
    Q.init();
    FOR(i,0,n) d[i]=-1;
    Q.push((point){a[1],0});
    FOR(i,L,n) {
        LL k=i;
        LL w=S[i];
        point t=Q.pop(k);
        d[i]=-k*t.x+t.y+w;
        int j=i-L+1;
        if (~d[j]) Q.push((point){a[j+1],j*a[j+1]-S[j]+d[j]});
    }
    printf("%lld\n", d[n]);
}

int main()
{

    while (scanf("%d%d", &n,&L)==2) {
        FOR(i,1,n) scanf("%lld", &a[i]);
        sort(a+1,a+1+n);
        FOR(i,1,n) S[i]=a[i]+S[i-1];
        solve();
    }

    return 0;
}
\end{lstlisting}

\section{区间DP}

四边形不等式优化

\begin{lstlisting}
for (int i=0; i<=n; i++) {
    g[i][i]=i;
}
for (int len=1; len<=n; len++) {
    for (int i=1; i<=n-len; i++) {
        int j=i+len;
        f[i][j]=INF;
        for (int k=g[i][j-1]; k<=g[i+1][j]; k++) {
            if (f[i][j]>f[i][k]+f[k+1][j]+w[i][j]) {
                f[i][j]=f[i][k]+f[k+1][j]+w[i][j];
                g[i][j]=k;
            }
        }
    }
}
\end{lstlisting}

Palindrome subsequence HDU - 4632

求回文子序列个数

经典

f[i][j]=f[i+1][j]+f[i][j-1]-f[i+1][j-1],由容斥原理得到

if (s[i]==s[j]) f[i][j]+=f[i+1][j-1]+1,如果s[i]==s[j],说明区间[i+1,j-1]所有子序列加上i和j又可以得到新的子序列,个数为f[i+1][j-1],还有加上1个s[i]和s[j]构成的子序列。

注意求模时产生负数的处理

\begin{lstlisting}
int n,f[MAXN][MAXN];
char s[MAXN];

int main()
{

    int T; scanf("%d", &T);
    FOR(kase,1,T) {
        scanf("%s", s+1);
        n=strlen(s+1);
        FOR(i,1,n) f[i][i]=1;
        FOR(len,1,n)FOR(i,1,n-len) {
            int j=i+len;
            f[i][j]=f[i][j-1]+f[i+1][j]-f[i+1][j-1];
            if (s[i]==s[j]) f[i][j]+=f[i+1][j-1]+1;
            f[i][j]%=MOD;
        }
        printf("Case %d: %d\n", kase,(f[1][n]+MOD)%MOD);
    }

    return 0;
}
\end{lstlisting}

Brackets POJ - 2955

括号匹配

经典

给一个括号组成的序列,问最长的合法括号子序列有多长

如果序列两端是一对括号,那么方程可以由去掉这对括号的子序列转移而来,并且还有以下转移:d[i][j]=d[i][k]+d[k+1][j]

\begin{lstlisting}
int main()
{

    while (scanf("%s", s+1)==1 && s[1]!='e') {
        n=strlen(s+1);
        FOR(i,1,n) {
            switch (s[i]) {
                case '(': a[i]=-2; break;
                case '[': a[i]=-1; break;
                case ']': a[i]=1; break;
                case ')': a[i]=2; break;
            }
        }
        FOR(len,1,n)FOR(i,1,n-len) {
            int j=i+len;
            d[i][j]=d[i+1][j-1];
            if (a[i]<0 && a[i]+a[j]==0) d[i][j]+=2;
            FOR(k,i,j) d[i][j]=max(d[i][j], d[i][k]+d[k+1][j]);
        }
        printf("%d\n", d[1][n]);
    }

    return 0;
}
\end{lstlisting}