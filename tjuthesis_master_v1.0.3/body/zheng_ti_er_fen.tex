\section{整体二分}

P3834 【模板】可持久化线段树 1(主席树)

静态区间第k大 整体二分 模板

巨坑爹的地方是,只能用(l+r)>>1而不能用(l+r)/2,因为有负数,导致取整方向不对,导致二分死循环。

\begin{lstlisting}
struct query {
    int l,r,k,op,ind;
} q[MAXN<<1],q1[MAXN<<1],q2[MAXN<<1];
struct BIT {
    int n,tree[MAXN];
    void init(int n){ this->n=n; memset(tree,0,sizeof(tree)); }
    void add(int x, int d){ for (; x<=n; x+=x&-x) tree[x]+=d; }
    int sum(int x){ int ans=0; for (; x; x-=x&-x) ans+=tree[x]; return ans; }
} bit;
int n,m,res[MAXN];
void solve(int l, int r, int ql, int qr) {
    if (ql>qr) return;
    if (l==r) {
        FOR(i,ql,qr) if (q[i].op==2) res[q[i].ind]=l;
        return;
    }
    int mid=(l+r)>>1,cnt1=0,cnt2=0;
    FOR(i,ql,qr) {
        if (q[i].op==1) {
            if (q[i].l<=mid) q1[++cnt1]=q[i], bit.add(q[i].ind,q[i].r);
            else q2[++cnt2]=q[i];
        } else {
            int x=bit.sum(q[i].r)-bit.sum(q[i].l-1);
            if (q[i].k<=x) q1[++cnt1]=q[i];
            else q[i].k-=x, q2[++cnt2]=q[i];
        }
    }
    FOR(i,1,cnt1) if (q1[i].op==1) bit.add(q1[i].ind,-q[i].r);
    FOR(i,1,cnt1) q[ql+i-1]=q1[i];
    FOR(i,1,cnt2) q[ql+i-1+cnt1]=q2[i];
    solve(l,mid,ql,ql+cnt1-1);
    solve(mid+1,r,ql+cnt1,qr);
}

int main()
{


    while (scanf("%d%d", &n,&m)==2) {
        FOR(i,1,n) {
            int x; scanf("%d", &x);
            q[i]={x,0,0,1,i};
        }
        FOR(i,1,m) {
            int l,r,k; scanf("%d%d%d", &l,&r,&k);
            q[n+i]={l,r,k,2,i};
        }
        bit.init(n);
        solve(-INF,INF,1,n+m);
        FOR(i,1,m) printf("%d\n", res[i]);
    }

    return 0;
}
\end{lstlisting}

P2617 Dynamic Rankings

单点修改 区间第k大 整体二分 模板

不吸氧比吸氧的带修主席树快一倍,吸氧用时是吸氧的带修主席树的1/5

\begin{lstlisting}
struct query {
    int l,r,k,ind,op;
} q[MAXN*3],q1[MAXN*3],q2[MAXN*3];
struct BIT {
    int n,tree[MAXN];
    void init(int n){ this->n=n; memset(tree,0,sizeof(tree)); }
    void add(int x, int d){ for (; x<=n; x+=x&-x) tree[x]+=d; }
    int sum(int x){ int ans=0; for (; x; x-=x&-x) ans+=tree[x]; return ans; }
} bit;
int n,m,a[MAXN],cnt,tot,res[MAXN];
void solve(int l, int r, int ql, int qr) {
    if (ql>qr) return;
    if (l==r) {
        FOR(i,ql,qr) if (q[i].op==2) res[q[i].ind]=l;
        return;
    }
    int mid=(l+r)>>1,cnt1=0,cnt2=0;
    FOR(i,ql,qr) {
        if (q[i].op==1) {
            if (q[i].l<=mid) q1[++cnt1]=q[i], bit.add(q[i].ind,q[i].r);
            else q2[++cnt2]=q[i];
        } else {
            int x=bit.sum(q[i].r)-bit.sum(q[i].l-1);
            if (q[i].k<=x) q1[++cnt1]=q[i];
            else q[i].k-=x, q2[++cnt2]=q[i];
        }
    }
    FOR(i,1,cnt1) if (q1[i].op==1) bit.add(q1[i].ind,-q1[i].r);
    FOR(i,1,cnt1) q[ql+i-1]=q1[i];
    FOR(i,1,cnt2) q[ql+i-1+cnt1]=q2[i];
    solve(l,mid,ql,ql+cnt1-1);
    solve(mid+1,r,ql+cnt1,qr);
}

int main()
{

    while (scanf("%d%d", &n,&m)==2) {
        cnt=tot=0;
        FOR(i,1,n) {
            scanf("%d", &a[i]);
            q[++cnt]={a[i],1,0,i,1};
        }
        FOR(i,1,m) {
            char op[2]; scanf("%s", op);
            if (op[0]=='Q') {
                int l,r,k; scanf("%d%d%d", &l,&r,&k);
                q[++cnt]={l,r,k,++tot,2};
            } else {
                int p,x; scanf("%d%d", &p,&x);
                q[++cnt]={a[p],-1,0,p,1};
                q[++cnt]={a[p]=x,1,0,p,1};
            }
        }
        bit.init(n);
        solve(-INF,INF,1,cnt);
        FOR(i,1,tot) printf("%d\n", res[i]);
    }

    return 0;
}
\end{lstlisting}

P3332 [ZJOI2013]K大数查询

区间加 区间第k大 整体二分 模板

题意:区间加上一个数,不是累加,是添一个数

\begin{lstlisting}
namespace ST {
    #define ls x<<1
    #define rs x<<1|1
    #define mid ((l+r)>>1)
    struct node {
        int l,r;
        LL sum,add;
        void update(LL val) {
            sum+=val*(r-l+1);
            add+=val;
        }
    } tree[MAXN<<2];
    void down(int x) {
        if (tree[x].add) {
            tree[ls].update(tree[x].add);
            tree[rs].update(tree[x].add);
            tree[x].add=0;
        }
    }
    void up(int x) {
        tree[x].sum=tree[ls].sum+tree[rs].sum;
    }
    void build(int x, int l, int r) {
        tree[x]={l,r,0,0};
        if (l<r) {
            build(ls,l,mid);
            build(rs,mid+1,r);
        }
    }
    void update(int x, int l, int r, int ql, int qr, LL val) {
        if (ql<=l && r<=qr) tree[x].update(val);
        else {
            down(x);
            if (ql<=mid) update(ls,l,mid,ql,qr,val);
            if (mid<qr) update(rs,mid+1,r,ql,qr,val);
            up(x);
        }
    }
    LL query(int x, int l, int r, int ql, int qr) {
        if (ql<=l && r<=qr) return tree[x].sum;
        else {
            LL ans=0;
            down(x);
            if (ql<=mid) ans+=query(ls,l,mid,ql,qr);
            if (mid<qr) ans+=query(rs,mid+1,r,ql,qr);
            up(x);
            return ans;
        }
    }
    #undef mid
}
struct query {
    int l,r;
    LL k;
    int ind,op;
} q[MAXN],q1[MAXN],q2[MAXN];
int n,m,tot,res[MAXN];
void solve(int l, int r, int ql, int qr) {
    if (ql>qr) return;
    if (l==r) {
        FOR(i,ql,qr) if (q[i].op==2) res[q[i].ind]=l;
        return;
    }
    int mid=(l+r)>>1,cnt1=0,cnt2=0;
    FOR(i,ql,qr) {
        if (q[i].op==1) {
            if (q[i].k<=mid) q1[++cnt1]=q[i];
            else q2[++cnt2]=q[i], ST::update(1,1,n,q[i].l,q[i].r,1);
        } else {
            LL x=ST::query(1,1,n,q[i].l,q[i].r);
            if (q[i].k<=x) q2[++cnt2]=q[i];
            else q[i].k-=x, q1[++cnt1]=q[i];
        }
    }
    FOR(i,1,cnt2) if (q2[i].op==1) ST::update(1,1,n,q2[i].l,q2[i].r,-1);
    FOR(i,1,cnt1) q[ql+i-1]=q1[i];
    FOR(i,1,cnt2) q[ql+i-1+cnt1]=q2[i];
    solve(l,mid,ql,ql+cnt1-1);
    solve(mid+1,r,ql+cnt1,qr);
}

int main()
{

    while (scanf("%d%d", &n,&m)==2) {
        tot=0;
        FOR(i,1,m) {
            int op,l,r;
            LL x; scanf("%d%d%d%lld", &op,&l,&r,&x);
            q[i]={l,r,x,0,op};
            if (op==2) q[i].ind=++tot;
        }
        ST::build(1,1,n);
        solve(-n,n,1,m);
        FOR(i,1,tot) printf("%d\n", res[i]);
    }

    return 0;
}
\end{lstlisting}