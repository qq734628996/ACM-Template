\section{spfa}

poj 2139 (acm题解里面)

poj 3259

spfa判负环,判断是否存在负环,加上vis数组判断是否遍历了图的所有结点

spfa判正环,d数组初始化为0

bfs版dfs已经够快,除非数据被针对,否则只能用dfs版spfa,一个简单而有效的改进是,所有结点进队次数大于T*(n+m)次以上即可判断负环,T根据实际情况一般取2,此算法牺牲了正确性,不过大多数情况下是正确的

\begin{lstlisting}
int T, n, m, w, d[MAXN], inq[MAXN], cnt[MAXN], kase, vis[MAXN];
vector<pii> E[MAXN];

bool spfa(int s)
{
    memset(d+1, 0x3f, sizeof(int) * n);
    memset(cnt+1, 0, sizeof(int) * n);
    queue<int> Q;
    Q.push(s); inq[s] = kase; d[s] = 0; vis[s] = kase;
    while (!Q.empty()) {
        int u = Q.front(); Q.pop();
        inq[u] = 0;
        REP(i,SZ(E[u])) {
            int v = E[u][i].fi, dist = E[u][i].se;
            if (d[v] > d[u] + dist) {
                d[v] = d[u] + dist;
                if (inq[v] != kase) {
                    vis[v] = kase;
                    inq[v] = kase;
                    Q.push(v);
                    if (++cnt[v] > n) return false;
                }
            }
        }
    }
    return true;
}

int main()
{

    scanf("%d", &T);
    for (kase = 1; kase <= T; kase++) {
        scanf("%d%d%d", &n, &m, &w);
        FOR(i,1,n) E[i].clear();
        REP(i,m) {
            int x, y, z; scanf("%d%d%d", &x, &y, &z);
            E[x].pb(pii(y,z)); E[y].pb(pii(x,z));
        }
        REP(i,w) {
            int x, y, z; scanf("%d%d%d", &x, &y, &z);
            E[x].pb(pii(y,-z));
        }
        bool ans = true;
        FOR(i,1,n) if (vis[i] != kase) ans &= spfa(i);
        puts(ans ? "NO" : "YES");
    }

    return 0;
}
\end{lstlisting}

POJ - 2240

判正环,spfa\_dfs版875ms,可能是边的存储慢了,或者数据被针对?或者dfs版写错了?还是输入或者map的问题?

\begin{lstlisting}
int n, m, vis[MAXN], ins[MAXN], kase;
double E[MAXN][MAXN], d[MAXN];
map<string, int> mp;

bool spfa_dfs(int u)
{
    vis[u] = ins[u] = kase;
    FOR(v,1,n) if (d[v] < d[u]*E[u][v]) {
        d[v] = d[u]*E[u][v];
        if (ins[v] == kase || spfa_dfs(v)) return true;
    }
    ins[u] = 0;
    return false;
}

int main()
{

    while (scanf("%d", &n), n) {
        printf("Case %d: ", ++kase);
        mp.clear();
        FOR(i,1,n) {
            string s; cin >> s;
            mp[s] = i;
        }
        scanf("%d", &m);
        memset(E, 0, sizeof(E));
        FOR(i,1,m) {
            string s, t; double r; cin >> s >> r >> t;
            E[mp[s]][mp[t]] = r;
        }
        bool ok = false;
        FOR(i,1,n) if (vis[i] != kase) {
            FOR(i,1,n) d[i] = 0;
            d[i] = 1;
            ok |= spfa_dfs(i);
            if (ok) break;
        }
        puts(ok ? "Yes" : "No");
    }

    return 0;
}
\end{lstlisting}