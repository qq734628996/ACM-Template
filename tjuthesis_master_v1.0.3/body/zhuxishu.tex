\section{主席树}

P3834 【模板】可持久化线段树 1(主席树)

区间第k大 裸题

\begin{lstlisting}
#define mid (l+r)/2
#define lt T[x].ls
#define rt T[x].rs
struct node {
    int sz,ls,rs;
} T[MAXN*20];
int root[MAXN],tot;
void init(){ tot=0; root[0]=0; T[0]={0,0,0}; }
int cp(int x){ T[++tot]=T[x]; return tot; }
void update(int& x, int v, int l, int r, int val) {
    x=cp(v);
    T[x].sz++;
    if (l<r) {
        if (val<=mid) update(lt,lt,l,mid,val);
        else update(rt,rt,mid+1,r,val);
    }
}
int query(int u, int v, int l, int r, int k) {
    if (l>=r) return l;
    int sz=T[T[v].ls].sz-T[T[u].ls].sz;
    if (k<=sz) return query(T[u].ls,T[v].ls,l,mid,k);
    else return query(T[u].rs,T[v].rs,mid+1,r,k-sz);
}
int n,m,q,a[MAXN],b[MAXN];
int ID(int x){ return lower_bound(b+1,b+1+m,x)-b; }

int main()
{

    while (scanf("%d%d", &n,&q)==2) {
        init();
        FOR(i,1,n) scanf("%d", &a[i]), b[i]=a[i];
        sort(b+1,b+1+n);
        m=unique(b+1,b+1+n)-b-1;
        FOR(i,1,n) update(root[i],root[i-1],1,m,ID(a[i]));
        while (q--) {
            int l,r,k; scanf("%d%d%d", &l,&r,&k);
            printf("%d\n", b[query(root[l-1],root[r],1,m,k)]);
        }
    }

    return 0;
}
\end{lstlisting}

K-th Closest Distance HDU - 6621

区间小于数x的个数 模板

\begin{lstlisting}
//the number of x s.t. x<k
int query(int u, int v, int l, int r, int k) {
    if (k<=l) return 0;
    if (k>r) return sum[v]-sum[u];
    return query(L[u],L[v],l,mid,k)+query(R[u],R[v],mid+1,r,k);
}
\end{lstlisting}

P3919 【模板】可持久化数组(可持久化线段树/平衡树)

可持久化线段树 单点修改 单点查询

\begin{lstlisting}
#define mid (l+r)/2
#define lt T[x].ls
#define rt T[x].rs
struct node {
    int val,ls,rs;
} T[MAXN<<5];
int root[MAXN],tot;
void init(){ tot=0; }
int newnode(int val){ T[++tot]={val,0,0}; return tot; }
int cp(int x){ T[++tot]=T[x]; return tot; }
int build(int l, int r) {
    int x=newnode(0);
    if (l==r) scanf("%d", &T[x].val);
    else {
        lt=build(l,mid);
        rt=build(mid+1,r);
    }
    return x;
}
int update(int v, int l, int r, int p, int val) {
    int x=cp(v);
    if (l==r) T[x].val=val;
    else {
        if (p<=mid) lt=update(lt,l,mid,p,val);
        else rt=update(rt,mid+1,r,p,val);
    }
    return x;
}
int query(int x, int l, int r, int p) {
    if (l==r) return T[x].val;
    else {
        if (p<=mid) return query(lt,l,mid,p);
        else return query(rt,mid+1,r,p);
    }
}

int main()
{

    int n,q;
    while (scanf("%d%d", &n,&q)==2) {
        init();
        root[0]=build(1,n);
        FOR(i,1,q) {
            int v,op,p,val; scanf("%d%d%d", &v,&op,&p);
            if (op==1) {
                scanf("%d", &val);
                root[i]=update(root[v],1,n,p,val);
            } else {
                root[i]=root[v];
                printf("%d\n", query(root[v],1,n,p));
            }
        }
    }

    return 0;
}
\end{lstlisting}

P3567 [POI2014]KUR-Couriers

查询区间中出现超过区间长度一半的数,没有输出0

当且仅当一个结点size超过区间长度一半才有答案,查询左右孩子size,递归下去即可

(如果查询最小,查询最大,查询中间大,貌似会超时,常数大)

\begin{lstlisting}
int query(int u, int v, int l, int r, int len) {
    if (l==r) return l;
    int lsz=T[T[v].ls].sz-T[T[u].ls].sz;
    int rsz=T[T[v].rs].sz-T[T[u].rs].sz;
    if (2*lsz>len) return query(T[u].ls,T[v].ls,l,mid,len);
    if (2*rsz>len) return query(T[u].rs,T[v].rs,mid+1,r,len);
    return 0;
}
\end{lstlisting}

P2633 Count on a tree

树上第k大 模板

从根到某个结点可以看成一个区间,这样可以转化为区间第k大,方法是sum[u]+sum[v]-sum[lca]-sum[lca\_fa]

需要求出lca,用了树链剖分(常数小),而且dfs的时候顺便更新了主席树

\begin{lstlisting}
int n,m,q,a[MAXN],b[MAXN];
int to[MAXM],nxt[MAXM],f[MAXN],tot;
void add(int u, int v) {
    to[tot]=v;
    nxt[tot]=f[u];
    f[u]=tot++;
}
#define mid (l+r)/2
int sum[MAXN<<5],L[MAXN<<5],R[MAXN<<5],root[MAXN],cnt;
int cp(int x){ sum[++cnt]=sum[x]; L[cnt]=L[x]; R[cnt]=R[x]; return cnt; }
void update(int& x, int v, int l, int r, int val) {
    x=cp(v); sum[x]++;
    if (l<r) {
        if (val<=mid) update(L[x],L[x],l,mid,val);
        else update(R[x],R[x],mid+1,r,val);
    }
}
int ID(int x){ return lower_bound(b+1,b+1+m,x)-b; }
int sz[MAXN],d[MAXN],fa[MAXN],son[MAXN],top[MAXN];
void dfs(int u) {
    sz[u]=1; d[u]=d[fa[u]]+1; son[u]=0;
    update(root[u],root[fa[u]],1,m,ID(a[u]));
    for (int i=f[u]; ~i; i=nxt[i]) {
        int v=to[i];
        if (v==fa[u]) continue;
        fa[v]=u; dfs(v);
        sz[u]+=sz[v];
        if (sz[v]>sz[son[u]]) son[u]=v;
    }
}
void dfs(int u, int tp) {
    top[u]=tp;
    if (son[u]) dfs(son[u],tp);
    for (int i=f[u]; ~i; i=nxt[i]) {
        int v=to[i];
        if (v==fa[u]||v==son[u]) continue;
        dfs(v,v);
    }
}
int LCA(int x, int y) {
    while (top[x]!=top[y]) d[top[x]]>=d[top[y]]?x=fa[top[x]]:y=fa[top[y]];
    return d[x]>=d[y]?y:x;
}
void init() {
    tot=0;
    memset(f,-1,sizeof(f));
    cnt=fa[1]=sz[0]=0;
}
int query(int u, int v, int lca, int lca_fa, int l, int r, int k) {
    if (l>=r) return l;
    int sz=sum[L[u]]+sum[L[v]]-sum[L[lca]]-sum[L[lca_fa]];
    if (k<=sz) return query(L[u],L[v],L[lca],L[lca_fa],l,mid,k);
    return query(R[u],R[v],R[lca],R[lca_fa],mid+1,r,k-sz);
}

int main()
{

    while (scanf("%d%d", &n,&q)==2) {
        init();
        FOR(i,1,n) scanf("%d", &a[i]), b[i]=a[i];
        sort(b+1,b+1+n);
        m=unique(b+1,b+1+n)-b-1;
        REP(i,n-1) {
            int x,y; scanf("%d%d", &x,&y);
            add(x,y); add(y,x);
        }
        dfs(1); dfs(1,1);
        int ans=0;
        while (q--) {
            int x,y,z; scanf("%d%d%d", &x,&y,&z);
            x^=ans;
            int lca=LCA(x,y);
            ans=b[query(root[x],root[y],root[lca],root[fa[lca]],1,m,z)];
            printf("%d\n", ans);
        }
    }

    return 0;
}
\end{lstlisting}

Sequence II HDU - 5919

主席树 区间不同数个数 区间不同数第一次出现位置第k个位置 强制在线 模板

假设一个区间出现了k个不同的数,每个数第一次出现的位置从小到大记为p[i],求p[(k+1)/2]

方法一(TLE):预处理出每个数上一个出现的位置pre[i],这样预处理出pre数组,求一个区间[l,r]不同数个数即求这个pre数组区间种小于l的数有多少个,对于求中位数,二分一个m,求区间[l,m]的不同数个数。复杂度为O(nlogn\^2),在hdu上会超时

方法二:只维护第一次出现的位置,逆序遍历,如果第一次出现,该位置加1,否则还要把之前出现的位置减1。这样相当于维护了n个版本的线段树,每次单点修改。查询一个区间有多少个不同数个数,即相当于版本root[l]的中小于等于r的个数,即区间求和,跟线段树求和一样。假如有num个不同,现在要查第(num+1)/2的位置,即第k大,方法跟主席树一样,比较左右两个子树的size递归遍历下去即可。复杂度为O(nlogn)

\begin{lstlisting}
int n,q,a[MAXN];
#define mid (l+r)/2
int sum[MAXN<<6],L[MAXN<<6],R[MAXN<<6],root[MAXN],tot;
void init(){ tot=0; }
int cp(int x){ sum[++tot]=sum[x]; L[tot]=L[x]; R[tot]=R[x]; return tot; }
void update(int& x, int v, int l, int r, int p, int d) {
    x=cp(v); sum[x]+=d;
    if (l<r) {
        if (p<=mid) update(L[x],L[x],l,mid,p,d);
        else update(R[x],R[x],mid+1,r,p,d);
    }
}
int query_sum(int x, int l, int r, int ll, int rr) {
    if (ll<=l && r<=rr) return sum[x];
    int ans=0;
    if (ll<=mid) ans+=query_sum(L[x],l,mid,ll,rr);
    if (mid<rr) ans+=query_sum(R[x],mid+1,r,ll,rr);
    return ans;
}
int kth(int x, int l, int r, int k) {
    if (l==r) return l;
    int sz=sum[L[x]];
    if (k<=sz) return kth(L[x],l,mid,k);
    return kth(R[x],mid+1,r,k-sz);
}

int main()
{

    int T; scanf("%d", &T);
    FOR(kase,1,T) {
        init();
        printf("Case #%d:", kase);
        scanf("%d%d", &n,&q);
        FOR(i,1,n) scanf("%d", &a[i]);
        map<int,int> mp;
        root[n+1]=0;
        ROF(i,1,n) {
            update(root[i],root[i+1],1,n,i,1);
            if (mp.count(a[i])) update(root[i],root[i],1,n,mp[a[i]],-1);
            mp[a[i]]=i;
        }
        int ans=0;
        while (q--) {
            int l,r; scanf("%d%d", &l,&r);
            l=(l+ans)%n+1, r=(r+ans)%n+1;
            if (l>r) swap(l,r);
            int num=query_sum(root[l],1,n,l,r);
            ans=kth(root[l],1,n,(num+1)/2);
            printf(" %d", ans);
        }
        puts("");
    }

    return 0;
}
\end{lstlisting}

P3168 [CQOI2015]任务查询系统

差分 区间前k小数之和 强制在线 模板

给n个区间,每个区间有一个权值,问某个点所在所有区间中前k小权值之和为多少

容易想到差分,对于每个区间,在l点+1,在r+1点-1,询问前缀和,区间前k小权值和用主席树维护。

注意有一个容易错的地方,当有多个相同的数时,前k小不能仅仅用size判断了

主席树不用离散化的方法:l=1,r=1e18,r理论大于可能的最大值即可。观察update函数,由于是动态开点,每次update消耗log(r-l)的空间,共有n次update,所以空间复杂度是O(nlog(max\_r-min\_l)),所以r取足够大也不怕

\begin{lstlisting}
#define mid (l+r)/2
int sz[MAXN<<7],L[MAXN<<7],R[MAXN<<7],root[MAXN],tot;
LL sum[MAXN<<7];
void init(){ tot=0; }
int cp(int x){ sz[++tot]=sz[x]; L[tot]=L[x]; R[tot]=R[x]; sum[tot]=sum[x]; return tot; }
void update(int& x, int v, int l, int r, int val) {
    x=cp(v); sz[x]+=(val>0?1:-1); sum[x]+=val;
    if (l<r) {
        if (abs(val)<=mid) update(L[x],L[x],l,mid,val);
        else update(R[x],R[x],mid+1,r,val);
    }
}
LL query(int x, int l, int r, int k) {
    if (k<=0) return 0;
    if (sz[x]<=k) return sum[x];
    if (l==r) return (LL)k*l;
    return query(L[x],l,mid,k)+query(R[x],mid+1,r,k-sz[L[x]]);
}

int main()
{

    init();
    int n,q; scanf("%d%d", &n,&q);
    map<int,vector<pii> > mp;
    FOR(i,1,n) {
        int l,r,val; scanf("%d%d%d", &l,&r,&val);
        mp[l].pb({l,val});
        mp[r+1].pb({r+1,-val});
    }
    FOR(i,1,n) {
        root[i]=cp(root[i-1]);
        for (auto& j:mp[i]) {
            update(root[i],root[i],1,1e7,j.se);
        }
    }
    LL ans=1;
    while (q--) {
        int x,a,b,c; scanf("%d%d%d%d", &x,&a,&b,&c);
        ans=query(root[x],1,1e7,1+(a*ans+b)%c);
        printf("%lld\n", ans);
    }

    return 0;
}
\end{lstlisting}