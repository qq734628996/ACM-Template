\section{平衡树fhq\_treap}

区间加法

\begin{lstlisting}
struct fhq_treap {
    #define lt T[x].ls
    #define rt T[x].rs
    int root,tot,tmp;
    struct node {
        int rnd,sz,ls,rs;
        LL val,sum,add;
    } T[MAXN];
    void build(int& x, LL val){ T[x=++tot]={rand(),1,0,0,val,val}; }
    void addone(int x, LL val) {
        if (x) {
            T[x].val+=val; T[x].add+=val;
            T[x].sum+=T[x].sz*val;
        }
    }
    void init() {
        srand(19981111);
        root=tot=0;
        T[0]={INF};
    }
    void up(int x) {
        if (x) {
            T[x].sz=T[lt].sz+T[rt].sz+1;
            T[x].sum=T[lt].sum+T[rt].sum+T[x].val;
        }
    }
    void down(int x) {
        if (x) {
            if (T[x].add) addone(lt,T[x].add), addone(rt,T[x].add);
            T[x].add=0;
        }
    }
    void Merge(int& x, int l, int r) {
        if (!l || !r) x=l+r;
        else if (T[l].rnd<T[r].rnd) down(x=l), Merge(rt,rt,r), up(x);
        else down(x=r), Merge(lt,l,lt), up(x);
    }
    void build(int n) {
        REP(i,n) {
            LL x; scanf("%lld", &x);
            build(tmp,x); Merge(root,root,tmp);
        }
    }
    void split(int x, int& l, int& r, int k) {
        if (!x) { l=r=0; return; }
        if (k<=T[lt].sz) down(r=x), split(lt,l,lt,k), up(x);
        else down(l=x), split(rt,rt,r,k-T[lt].sz-1), up(x);
    }
    void add(int l, int r, LL val) {
        int x,y,z;
        split(root,x,y,r); split(x,x,z,l-1);
        addone(z,val);
        Merge(x,x,z); Merge(root,x,y);
    }
    LL query(int l, int r) {
        int x,y,z;
        split(root,x,y,r); split(x,x,z,l-1);
        LL ans=T[z].sum;
        Merge(x,x,z); Merge(root,x,y);
        return ans;
    }
} solve;
\end{lstlisting}

Tyvj 1728 普通平衡树 BZOJ- 3224 588ms

经典:

1. 插入x数

2. 删除x数(若有多个相同的数,因只删除一个)

3. 查询x数的排名(若有多个相同的数,因输出最小的排名)

4. 查询排名为x的数

5. 求x的前驱(前驱定义为小于x,且最大的数)

6. 求x的后继(后继定义为大于x,且最小的数)

\begin{lstlisting}
struct fhq_treap {
    #define lt T[x].ls
    #define rt T[x].rs
    int root,tot,tmp;
    struct node {
        int rnd,sum,sz,ls,rs;
    } T[MAXN];
    void init()
    {
        srand(19981111);
        root = tot = 0;
        T[0].rnd = T[0].sum = INF;
    }
    void build(int& x, int val) { T[x=++tot] = node{rand()<<15|rand(),val,1}; }
    void up(int x) { if (x) T[x].sz = T[lt].sz+T[rt].sz+1; }
    void Merge(int& x, int l, int r)
    {
        if (!l || !r) x = l+r;
        else if (T[l].rnd<T[r].rnd) x=l, Merge(rt,rt,r), up(x);
        else x=r, Merge(lt,l,lt), up(x);
    }
    void split(int x, int& l, int& r, int k)
    {
        if (!k) l=0, r=x;
        else if (k==T[x].sz) l=x, r=0;
        else if (k<=T[lt].sz) r=x, split(lt,l,lt,k), up(x);
        else l=x, split(rt,rt,r,k-T[lt].sz-1), up(x);
    }
    int Rank(int x, int val)
    {
        if (!x) return 0;
        if (T[x].sum>=val) return Rank(lt,val);
        return T[lt].sz+1+Rank(rt,val);
    }
    void Insert(int val)
    {
        int x,y,rk = Rank(root,val);
        split(root,x,y,rk);
        build(tmp,val);
        Merge(x,x,tmp); Merge(root,x,y);
    }
    void del(int val)
    {
        int x,y,z,rk = Rank(root,val);
        split(root,x,y,rk+1); split(x,x,z,rk);
        Merge(root,x,y);
    }
    int Kth(int k)
    {
        int x,y,z;
        split(root,x,y,k); split(x,x,z,k-1);
        int ans = T[z].sum;
        Merge(x,x,z); Merge(root,x,y);
        return ans;
    }
    int pre(int val)
    {
        int x,y,z,rk = Rank(root,val);
        split(root,x,y,rk); split(x,x,z,rk-1);
        int ans = T[z].sum;
        Merge(x,x,z); Merge(root,x,y);
        return ans;
    }
    int succ(int val)
    {
        int x,y,z,rk = Rank(root,val+1);
        split(root,x,y,rk+1); split(x,x,z,rk);
        int ans = T[z].sum;
        Merge(x,x,z); Merge(root,x,y);
        return ans;
    }
};

fhq_treap solve;
int n,op,x;

int main()
{

    scanf("%d", &n);
    solve.init();
    while (n--) {
        scanf("%d%d", &op,&x);
        switch (op) {
        case 1: solve.Insert(x);
            break;
        case 2: solve.del(x);
            break;
        case 3: printf("%d\n", solve.Rank(solve.root,x)+1);
            break;
        case 4: printf("%d\n", solve.Kth(x));
            break;
        case 5: printf("%d\n", solve.pre(x));
            break;
        case 6: printf("%d\n", solve.succ(x));
            break;
        }
    }

    return 0;
}
\end{lstlisting}

SuperMemo POJ - 3580

区间加、翻转、旋转,插入、删除,求最值

加了push\_up、push\_down

模板 891ms

\begin{lstlisting}
struct fhq_treap {
    #define lt T[x].ls
    #define rt T[x].rs
    int root,tot,tmp;
    struct node {
        int rnd,sum,mi,sz,add,rev,ls,rs;
    } T[MAXN];
    void build(int& x, int val) { T[x=++tot] = node{rand()<<15|rand(),val,val,1}; }
    void addone(int x, int val) { if (x) T[x].sum+=val, T[x].mi+=val, T[x].add+=val; }
    void revone(int x) { T[x].rev ^= 1, swap(lt,rt); }
    void init()
    {
        srand(19981111);
        root = tot = 0;
        T[0] = node{INF,INF,INF};
    }
    void up(int x)
    {
        if (x) {
            T[x].mi = min(T[x].sum, min(T[lt].mi, T[rt].mi));
            T[x].sz = T[lt].sz + T[rt].sz + 1;
        }
    }
    void down(int x)
    {
        if (x) {
            if (T[x].add) addone(lt,T[x].add), addone(rt,T[x].add);
            if (T[x].rev) revone(lt), revone(rt);
            T[x].add = T[x].rev = 0;
        }
    }
    void Merge(int& x, int l, int r)
    {
        if (!l || !r) x=l+r;
        else if (T[l].rnd<T[r].rnd) down(x=l), Merge(rt,rt,r), up(x);
        else down(x=r), Merge(lt,l,lt), up(x);
    }
    void build(int n)
    {
        while (n--) {
            int x; scanf("%d", &x);
            build(tmp,x); Merge(root,root,tmp);
        }
    }
    void split(int x, int& l, int& r, int k)
    {
        if (!k) l=0, r=x;
        else if (k==T[x].sz) l=x, r=0;
        else if (k<=T[lt].sz) down(r=x), split(lt,l,lt,k), up(x);
        else down(l=x), split(rt,rt,r,k-T[lt].sz-1), up(x);
    }
    void Insert(int pos, int val)
    {
        int x,y;
        split(root,x,y,pos);
        build(tmp,val);
        Merge(x,x,tmp), Merge(root,x,y);
    }
    void del(int pos)
    {
        int x,y,z;
        split(root,x,z,pos), split(x,x,y,pos-1);
        Merge(root,x,z);
    }
    void add(int l, int r, int val)
    {
        int x,y,z;
        split(root,x,z,r), split(x,x,y,l-1);
        addone(y,val);
        Merge(x,x,y), Merge(root,x,z);
    }
    void Reverse(int l, int r)
    {
        int x,y,z;
        split(root,x,z,r), split(x,x,y,l-1);
        revone(y);
        Merge(x,x,y), Merge(root,x,z);
    }
    void revole(int l, int r, int k)
    {
        k %= r-l+1;
        int x,y,z,h;
        split(root,x,h,r), split(x,x,z,r-k), split(x,x,y,l-1);
        Merge(x,x,z), Merge(x,x,y), Merge(root,x,h);
    }
    int query_mi(int l, int r)
    {
        int x,y,z;
        split(root,x,z,r), split(x,x,y,l-1);
        int ans = T[y].mi;
        Merge(x,x,y), Merge(root,x,z);
        return ans;
    }
};

fhq_treap solve;
int n,m,x,y,z;
char op[10];

int main()
{

    scanf("%d", &n);
    solve.init();
    solve.build(n);
    scanf("%d", &m);
    while (m--) {
        scanf("%s", op);
        switch (op[0]) {
        case 'A': scanf("%d%d%d", &x,&y,&z), solve.add(x,y,z);
            break;
        case 'R': scanf("%d%d", &x,&y);
            if (op[3] == 'E') solve.Reverse(x,y);
            else scanf("%d", &z), solve.revole(x,y,z);
            break;
        case 'I': scanf("%d%d", &x,&y), solve.Insert(x,y);
            break;
        case 'D': scanf("%d", &x), solve.del(x);
            break;
        case 'M': scanf("%d%d", &x,&y), printf("%d\n", solve.query_mi(x,y));
            break;
        }
    }

    return 0;
}
\end{lstlisting}

递归打印全体(中序遍历)

\begin{lstlisting}
void print(int x)
{
	if (x) {
		down(x);
		print(lt);
		printf("%d ", T[x].sum);
		print(rt);
	}
}
\end{lstlisting}

treap线性构树的方法:跟笛卡尔树一样

https://blog.sengxian.com/algorithms/treap

Version Controlled IDE UVA - 12538

线性建树、pos插入k个、pos删除k个、pos后打印k个、强制在线

线性建树与笛卡尔树建树一样,用一个栈保存最右边一条链。可持久化:fhq\_treap由于不修改结点,故每次操作都新建一个结点,并把每次操作的根保存,根的编号即是版本号。此处还需要用到内存池保存结点——重载new运算符。

一直RE,把rand()<<15|rand()改成rand()就好了!!!???

100ms,优秀

\begin{lstlisting}
int tot,cntC;
struct node *null,*pit;
struct node {
    int key,rnd,sz;
    node *l,*r;
    node(){}
    node(int key, int rnd = rand()):key(key),rnd(rnd),sz(1){ l=r=null; }
    void* operator new(size_t){ return pit++; }
    void upd(){ sz = l->sz+r->sz+1; }
} pool[4000010], *root[MAXN];
node* newnode(node* o) { return o==null ? o : new node{*o}; }
node* Merge(node* a, node* b)
{
    if (a==null) return newnode(b);
    if (b==null) return newnode(a);
    node* tmp;
    if (a->rnd < b->rnd) {
        tmp = newnode(a);
        tmp->r = Merge(a->r,b);
    } else {
        tmp = newnode(b);
        tmp->l = Merge(a,b->l);
    }
    tmp->upd();
    return tmp;
}
typedef pair<node*, node*> droot;
droot split(node* o, int k)
{
    droot d(null,null);
    if (o==null) return d;
    if (!k) return droot(null,newnode(o));
    if (k==o->sz) return droot(newnode(o),null);
    int sz = o->l->sz;
    node* tmp = newnode(o);
    if (k<=sz) {
        d = split(o->l,k);
        tmp->l = d.se;
        tmp->upd();
        d.se = tmp;
    } else {
        d = split(o->r,k-sz-1);
        tmp->r = d.fi;
        tmp->upd();
        d.fi = tmp;
    }
    return d;
}
node* stk[MAXN];
node* build(char* p)
{
    node* rt = new node(-INF,-INF);
    stk[0] = rt; int sz = 1;
    for (; *p; p++) {
        node* now = new node(*p); int top = sz-1;
        while (stk[top]->rnd > now->rnd) stk[top--]->upd();
        if (top != sz-1) now->l = stk[top+1];
        stk[top]->r = now; sz = top+1;
        stk[sz++] = now;
    }
    while (sz) stk[--sz]->upd();
    return rt->r;
}
void print(node* o)
{
    if (o==null) return;
    print(o->l);
    putchar(o->key);
    if (o->key == 'c') cntC++;
    print(o->r);
}
void init()
{
    srand(time(NULL));
    pit = pool; null = new node();
    null->sz = tot = cntC = 0;
    root[0] = null;
}
void Insert(int pos, char* s)
{
    droot l = split(root[tot],pos);
    root[++tot] = Merge(Merge(l.fi,build(s)),l.se);
}
void Remove(int pos, int k)
{
    droot l = split(root[tot],pos-1);
    droot r = split(l.se,k);
    root[++tot] = Merge(l.fi,r.se);
}
void print(int v, int pos, int k)
{
    droot l = split(root[v],pos-1);
    droot r = split(l.se,k);
    print(r.fi); puts("");
}

int main()
{

    char s[MAXN];
    init();
    int n; scanf("%d", &n);
    REP(i,n) {
        int op; scanf("%d", &op);
        int v,p,k;
        if (op == 1) scanf("%d%s", &p,s), Insert(p-cntC,s);
        else if (op == 2) scanf("%d%d", &p,&k), Remove(p-cntC,k-cntC);
        else scanf("%d%d%d", &v,&p,&k), print(v-cntC,p-cntC,k-cntC);
    }

    return 0;
}
\end{lstlisting}