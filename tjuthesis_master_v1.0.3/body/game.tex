\section{博弈论}

https://www.bilibili.com/video/av9114486?from=search\&seid=1678403368497689776

Brave Game HDU - 1846

Bash博弈

有n个石子,每人最多取m个,轮流操作

n\%(m+1)

\begin{lstlisting}
int main()
{

    int T; scanf("%d", &T);
    while (T--) {
        int a,b; scanf("%d%d", &a,&b);
        if (a%(b+1)) puts("first");
        else puts("second");
    }

    return 0;
}
\end{lstlisting}

取石子游戏 HDU - 2516

斐波那契博弈裸题

n个石子轮流取,第一次可以取任意多个但不能全部取完,以后每次取石子不能超过上次取数的两倍,取完者胜

如果是斐波那契数,必败,否则必胜

\begin{lstlisting}
LL n,fib[55];

int main()
{

    fib[1]=1;
    FOR(i,2,54) fib[i]=fib[i-1]+fib[i-2];
    while (cin>>n && n) {
        bool ok=true;
        FOR(i,2,54) {
            if (fib[i]==n) ok=false;
            if (fib[i]>n) break;
        }
        puts(ok?"First win":"Second win");
    }

    return 0;
}
\end{lstlisting}

威佐夫博弈(Wythoff Game):两堆物品,两人轮流从某堆或者同时在两堆物品中取同样多的物品,至少一个,不能操作者输

奇异局势:必败的局势形如(ak,bk),ak为之前未出现过的最小自然数,bk=ak+k

判断奇异局势:(b-a)*(sqrt(5)+1)/2==a

取石子游戏 POJ - 1067

威佐夫博弈裸题

\begin{lstlisting}
int main()
{

    int a,b;
    while (scanf("%d%d", &a,&b)==2) {
        if (a>b) swap(a,b);
        int c=floor((b-a)*(sqrt(5)+1)/2);
        puts(a==c?"0":"1");
    }

    return 0;
}
\end{lstlisting}

Nim游戏:

奇异局势:异或和等于0

奇异局势的转换:假设异或和为sum,对于任意一个数ai,如果sum\^ai<ai,则可以拿走ai-(sum\^ai)个数。

Being a Good Boy in Spring Festival HDU - 1850

Nim游戏裸题

如果有解,输出第一步可行解的方案数

\begin{lstlisting}
int n,a[MAXN];

int main()
{

    while (scanf("%d", &n)==1 && n) {
        int ans=0,cnt=0;
        REP(i,n) scanf("%d", &a[i]), ans^=a[i];
        if (!ans) puts("0");
        else {
            REP(i,n) {
                int t=ans^a[i];
                if (t<a[i]) cnt++;
            }
            printf("%d\n", cnt);
        }
    }

    return 0;
}
\end{lstlisting}

Good Luck in CET-4 Everybody!HDU - 1847

SG函数

SG函数由记忆化搜索得到

\begin{lstlisting}
int n,sg[MAXN],a[22];
int mex(int i)
{
    if (~sg[i]) return sg[i];
    int vis[MAXN];
    memset(vis,0,sizeof(vis));
    REP(j,11) {
        int k=i-a[j];
        if (k<0) break;
        vis[mex(k)]=1;
    }
    REP(j,MAXN) if (!vis[j]) { sg[i]=j; break; }
    return sg[i];
}

int main()
{

    a[0]=1;
    FOR(i,1,10) a[i]=a[i-1]*2;
    memset(sg,-1,sizeof(sg));
    while (scanf("%d", &n)==1) {
        puts(mex(n)?"Kiki":"Cici");
    }

    return 0;
}
\end{lstlisting}

kuangbin博弈

Game of CS LightOJ - 1355

green博弈变形

参考了网上题解

green博弈:给定一颗树,每次可以删掉一些边并且连带删掉子树,不能操作者输。此题稍有不同,每条边有一个权值,每次只能使权值减一,减为零即可删掉整个子树

结论:如果边权为1,即green博弈,sg(u)\^=sg(v)+1,如果边权大于1,只有是奇数的时候有贡献:sg(u)\^=sg(v)\^1

\begin{lstlisting}
int to[MAXN],dis[MAXN],f[MAXN],nxt[MAXN],tot;
void init()
{
    tot=0;
    memset(f,-1,sizeof(f));
}
void add(int u, int v, int w)
{
    to[tot]=v;
    dis[tot]=w;
    nxt[tot]=f[u];
    f[u]=tot++;
}
int n;
int sg(int u, int fa)
{
    int ans=0;
    for (int i=f[u]; ~i; i=nxt[i]) {
        int v=to[i], w=dis[i];
        if (v==fa) continue;
        if (w==1) ans^=sg(v,u)+1;
        else ans^=sg(v,u)^(w&1);
    }
    return ans;
}

int main()
{

    int T; scanf("%d", &T);
    FOR(kase,1,T) {
        init();
        scanf("%d", &n);
        REP(i,n-1) {
            int x,y,z; scanf("%d%d%d", &x,&y,&z);
            add(x,y,z);
            add(y,x,z);
        }
        printf("Case %d: %s\n", kase,sg(0,-1)?"Emily":"Jolly");
    }

    return 0;
}
\end{lstlisting}

Misere Nim LightOJ - 1253

反Nim博弈裸题

反Nim博弈:取到最后的石子的人输

如果全是1,当且仅当异或和为0必胜,否则异或和非0必胜。

\begin{lstlisting}
int a[MAXN];

int main()
{

    int T; scanf("%d", &T);
    FOR(kase,1,T) {
        int n; scanf("%d", &n);
        REP(i,n) scanf("%d", &a[i]);
        int sum=0;
        REP(i,n) sum^=a[i];
        bool flag=true;
        REP(i,n) if (a[i]!=1) flag=false;
        printf("Case %d: %s\n", kase,((flag&&!sum)||(!flag&&sum))?"Alice":"Bob");
    }

    return 0;
}
\end{lstlisting}

\subsection{组合游戏总结}

Nim游戏:简单来说可以是n个数,轮流操作,每次选择一个数减小或者删掉,不能操作者输。

1. 两个游戏者轮流操作

2. 游戏的状态集有限,并且不管双方怎么走,都不会再现以前出现过的状态。

3. 谁不能操作谁输。


相关术语:

1. 一个状态是必败状态当且仅当它的所有后继都是必败状态。

2. 一个状态是必胜状态当且仅当它至少有一个后继是必败状态。


暴力解法:根据状态集,dp

Ferguson游戏:Nim的特例?

Chomp!游戏:有一个m*n的棋盘,每次可以取走一个方块并拿走这个方块右边及上边的所有方块,拿到最后一块方块的人输。

结论:除非开局只有一个方块,否则先手必胜

证明:假设后手有必胜决策可以达到必胜状态,则先手必可以抢先后手达到这个必胜状态,矛盾。

约数游戏:有1~n个数,轮流拿一个数并且拿走拿走它的约数,拿到最后一个数的胜。

结论:类似Chomp!游戏,先手必胜。

Bouton定理:状态(x\_1,x\_2,x\_3)是必败状态当且仅当x\_1\^x\_2\^x\_3=0,这个结果也称Nim和,这个定理可以推广到n维状态。

先手胜操作步骤:异或结果最高位的1记为第k位,找一个第k位为1的数,把它修改(注意第k位变为0,数肯定减小了),使得异或结果为0

证明:归纳法

组合游戏的和:由k个组合游戏构成的新游戏。

SG函数:SG(x)=mex(S),S为状态x的后继状态集,mex函数表示不在"S所有元素的SG函数值"中的最小非负整数。

SG定理:SG(x)的值等于各子游戏的Nim和。