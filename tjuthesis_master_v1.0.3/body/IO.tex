\section{IO外挂}

适合读到文件末尾、负数

\begin{lstlisting}
namespace fastIO {
#define BUF_SIZE 100000
    bool IOerror = 0;
    inline char nc() {
        static char buf[BUF_SIZE], *p1 = buf + BUF_SIZE, *pend = buf + BUF_SIZE;
        if(p1 == pend) {
            p1 = buf;
            pend = buf + fread(buf, 1, BUF_SIZE, stdin);
            if(pend == p1) {
                IOerror = 1;
                return -1;
            }
        }
        return *p1++;
    }
    template <class T>
    inline bool read(T &ret) {
        char c;
        if (c=nc(),c==EOF)return 0;
        while(c!='-'&&(c<'0'||c>'9'))c=nc();
        int sgn =(c=='-')?-1:1;
        ret=(c=='-')?0:(c - '0');
        while(c=nc(),c>='0'&&c<='9') ret=ret*10+(c-'0');
        ret *= sgn;
        return 1;
    }
    template <class T>
    inline void print(T x) {
        if(x>9) print(x/10);
        putchar(x%10+'0');
    }
#undef BUF_SIZE
};
\end{lstlisting}

适用于正整数

\begin{lstlisting}
namespace IN {
    const int N=2e6+10;
    char str[N],*S=str,*T=str;
    inline char rdc()
    {
        if (S==T) {
            T=(S=str)+fread(str,1,N,stdin);
            if (S==T) return EOF; //exit(0);
        }
        return *S++;
    }
    inline int rd()
    {
        int x=0;
        char t=rdc();
        while (t>'9'||t<'0') t=rdc();
        while ('0'<=t&&t<='9') x=x*10+t-'0', t=rdc();
        return x;
    }
}
namespace OUT {
    const int N=2e6+10;
    char str[N],*T=str;
    inline void otc(char t)
    {
        if (T==str+N) {
            fwrite(str,1,N,stdout);
            T=str;
        }
        *T++=t;
    }
    inline void ot(int x)
    {
        if (x<10) otc('0'+x);
        else {
            ot(x/10);
            otc('0'+x%10);
        }
    }
    inline void otall()
    {
        fwrite(str,1,T-str,stdout);
    }
}
//example:
//int n=IN::rd();
//OUT::ot(n)
//OUT::otc('\n');
//OUT::otall();
\end{lstlisting}