\section{树套树}

https://blog.csdn.net/xyc1719/article/details/83614580(splay好板子)

P3380 【模板】二逼平衡树(树套树)

线段树套平衡树

支持以下操作:

1. 查询k在区间内的排名

2. 查询区间内排名为k的值

3. 修改某一位值上的数值

4. 查询k在区间内的前驱(前驱定义为严格小于x,且最大的数,若不存在输出-2147483647)

5. 查询k在区间内的后继(后继定义为严格大于x,且最小的数,若不存在输出2147483647)

线段树维护区间信息,平衡树维护值域信息

具体来说,线段树每个结点开一颗平衡树,这个平衡树能支持rank,kth,insert,del,pre,succ操作。

操作1,写一个函数lt,求区间中小于val的数有多少个,答案即为lt+1

操作2,适用操作1的lt函数进行二分,复杂度为O(logn\^3),树状数组套主席树能搞到O(logn\^2)

操作3,对包含这个位置的区间全部暴力修改,O(logn\^2)

操作4,5,递归求出前驱后继,然后合并答案

对 https://blog.csdn.net/xyc1719/article/details/83614580 代码简化的一些学习:线段树只涉及单点修改,不需要维护l,r信息,递归函数中带上l,r即可

\begin{lstlisting}
struct treap_node {
    int rnd,sz,ls,rs,sum;
} T[MAXN<<5];
int tot;
void init(){ srand(19981111); tot=0; }
namespace treap {
    #define lt T[x].ls
    #define rt T[x].rs
    int newnode(int val){ T[++tot]={rand(),1,0,0,val}; return tot; }
    void up(int x){ if (x) T[x].sz=T[lt].sz+T[rt].sz+1; }
    void split(int x, int& l, int& r, int k) {
        if (!x) { l=r=0; return; }
        if (k<=T[lt].sz) r=x, split(lt,l,lt,k), up(x);
        else l=x, split(rt,rt,r,k-T[lt].sz-1), up(x);
    }
    void Merge(int& x, int l, int r) {
        if (!l||!r) x=l+r;
        else if (T[l].rnd<T[r].rnd) x=l, Merge(rt,rt,r), up(x);
        else x=r, Merge(lt,l,lt), up(x);
    }
    int Rank(int x, int val) { //the number less than val
        if (!x) return 0;
        if (val<=T[x].sum) return Rank(lt,val);
        return T[lt].sz+1+Rank(rt,val);
    }
    void Insert(int& root, int val) {
        int x,y;
        split(root,x,y,Rank(root,val));
        Merge(x,x,newnode(val)); Merge(root,x,y);
    }
    void del(int& root, int val) {
        int x,y,z,rnk=Rank(root,val);
        split(root,x,y,rnk+1); split(x,x,z,rnk);
        Merge(root,x,y);
    }
    int pre(int& root, int val) {
        int x,y,z,rnk=Rank(root,val);
        if (rnk<1) return -2147483647;
        split(root,x,y,rnk); split(x,x,z,rnk-1);
        int ans=T[z].sum;
        Merge(x,x,z); Merge(root,x,y);
        return ans;
    }
    int succ(int& root, int val) {
        int x,y,z,rnk=Rank(root,val+1);
        if (rnk>=T[root].sz) return 2147483647;
        split(root,x,y,rnk+1); split(x,x,z,rnk);
        int ans=T[z].sum;
        Merge(x,x,z); Merge(root,x,y);
        return ans;
    }
    void show(int x) {
        if (x) {
            show(lt);
            printf("%d ", T[x].sum);
            show(rt);
        }
    }
    #undef lt
    #undef rt
}
int n,q,a[MAXN];
namespace tree {
    #define ls x<<1
    #define rs x<<1|1
    struct tree_node {
        int l,r,root;
    } tree[MAXN<<2];
    void build(int x, int l, int r) {
        tree[x].l=l, tree[x].r=r;
        tree[x].root=0;
        FOR(i,l,r) treap::Insert(tree[x].root,a[i]);
        if (l<r) {
            int mid=(l+r)/2;
            build(ls,l,mid);
            build(rs,mid+1,r);
        }
    }
    int lt(int x, int l, int r, int val) {
        int L=tree[x].l, R=tree[x].r;
        if (l<=L&&R<=r) return treap::Rank(tree[x].root,val);
        else {
            int mid=(L+R)/2;
            int ans=0;
            if (l<=mid) ans+=lt(ls,l,r,val);
            if (mid<r) ans+=lt(rs,l,r,val);
            return ans;
        }
    }
    int Rank(int l, int r, int val) {
        return lt(1,l,r,val)+1;
    }
    int kth(int l, int r, int k) {
        int ll=-1, rr=1e8+9, ans;
        while (ll<=rr) {
            int mm=(ll+rr)/2;
            if (lt(1,l,r,mm)>=k) ans=mm-1, rr=mm-1;
            else ll=mm+1;
        }
        return ans;
    }
    void modify(int x, int pos, int val) {
        treap::del(tree[x].root,a[pos]);
        treap::Insert(tree[x].root,val);
        int L=tree[x].l, R=tree[x].r;
        if (L<R) {
            int mid=(L+R)/2;
            if (pos<=mid) modify(ls,pos,val);
            else modify(rs,pos,val);
        }
    }
    void modify(int pos, int val) {
        modify(1,pos,val);
        a[pos]=val;
    }
    int pre(int x, int l, int r, int val) {
        int L=tree[x].l, R=tree[x].r;
        if (l<=L&&R<=r) return treap::pre(tree[x].root,val);
        else {
            int mid=(L+R)/2;
            int ans=-2147483647;
            if (l<=mid) ans=max(ans,pre(ls,l,r,val));
            if (mid<r) ans=max(ans,pre(rs,l,r,val));
            return ans;
        }
    }
    int pre(int l, int r, int val) { return pre(1,l,r,val); }
    int succ(int x, int l, int r, int val) {
        int L=tree[x].l, R=tree[x].r;
        if (l<=L&&R<=r) return treap::succ(tree[x].root,val);
        else {
            int mid=(L+R)/2;
            int ans=2147483647;
            if (l<=mid) ans=min(ans,succ(ls,l,r,val));
            if (mid<r) ans=min(ans,succ(rs,l,r,val));
            return ans;
        }
    }
    int succ(int l, int r, int val){ return succ(1,l,r,val); }
    void show(int x) {
        int L=tree[x].l, R=tree[x].r;
        printf("[%d,%d]: ", L,R);
        treap::show(tree[x].root); puts("");
        if (L<R) show(ls), show(rs);
    }
    #undef ls
    #undef rs
}
using namespace tree;

int main()
{

    while (scanf("%d%d", &n,&q)==2) {
        init();
        FOR(i,1,n) scanf("%d", &a[i]);
        build(1,1,n);
        while (q--) {
            int op,l,r,k; scanf("%d%d%d", &op,&l,&r);
            if (op!=3) scanf("%d", &k);
            switch (op) {
                case 1: printf("%d\n", Rank(l,r,k)); break;
                case 2: printf("%d\n", kth(l,r,k)); break;
                case 3: modify(l,r); break;
                case 4: printf("%d\n", pre(l,r,k)); break;
                case 5: printf("%d\n", succ(l,r,k)); break;
            }
        }
    }

    return 0;
}
\end{lstlisting}

bzoj1901洛谷P2617 Dynamic Rankings

树状数组套主席树 模板 动态第k大

支持单点修改,区间询问第k大,O(logn\^2)

常数略大,需要开O2优化才能过

\begin{lstlisting}
#define mid (l+r)/2
inline int lowbit(int x){ return x&-x; }
int sum[MAXN<<8],L[MAXN<<8],R[MAXN<<8];
int xx[MAXN],yy[MAXN],rt[MAXN],a[MAXN],b[MAXN<<1],ca[MAXN],cb[MAXN],cc[MAXN];
int n,q,m,tot,totx,toty;
void init(){ tot=totx=toty=0; }
int ID(int x){ return lower_bound(b+1,b+1+m,x)-b; }
int cp(int x){ sum[++tot]=sum[x]; L[tot]=L[x]; R[tot]=R[x]; return tot; }
void update(int& x, int v, int l, int r, int p, int d) {
    x=cp(v); sum[x]+=d;
    if (l<r) {
        if (p<=mid) update(L[x],L[x],l,mid,p,d);
        else update(R[x],R[x],mid+1,r,p,d);
    }
}
int query(int l, int r, int k) {
    if (l==r) return b[l];
    int sz=0;
    FOR(i,1,totx) sz-=sum[L[xx[i]]];
    FOR(i,1,toty) sz+=sum[L[yy[i]]];
    if (k<=sz) {
        FOR(i,1,totx) xx[i]=L[xx[i]];
        FOR(i,1,toty) yy[i]=L[yy[i]];
        return query(l,mid,k);
    } else {
        FOR(i,1,totx) xx[i]=R[xx[i]];
        FOR(i,1,toty) yy[i]=R[yy[i]];
        return query(mid+1,r,k-sz);
    }
}
void add(int x, int y) {
    for (int i=x; i<=n; i+=lowbit(i)) update(rt[i],rt[i],1,m,ID(a[x]),y);
}

int main()
{

    while (scanf("%d%d", &n,&q)==2) {
        init();
        FOR(i,1,n) scanf("%d", &a[i]), b[i]=a[i];
        m=n;
        FOR(i,1,q) {
            char op[2]; scanf("%s%d%d", op,&ca[i],&cb[i]);
            if (op[0]=='Q') scanf("%d", &cc[i]);
            else b[++m]=cb[i], cc[i]=0;
        }
        sort(b+1,b+1+m);
        m=unique(b+1,b+1+m)-b-1;
        FOR(i,1,n) add(i,1);
        FOR(i,1,q) {
            if (cc[i]) {
                totx=toty=0;
                for (int j=ca[i]-1; j; j-=lowbit(j)) xx[++totx]=rt[j];
                for (int j=cb[i]; j; j-=lowbit(j)) yy[++toty]=rt[j];
                printf("%d\n", query(1,m,cc[i]));
            } else add(ca[i],-1), a[ca[i]]=cb[i], add(ca[i],1);
        }
    }

    return 0;
}
\end{lstlisting}

P3157 [CQOI2011]动态逆序对

支持两种操作,打印当前序列逆序对数;删掉一个数(凡是相同的全部删掉)

方法一:

线段树套平衡树 在线

先求出总逆序对数,没删掉一个数,假设该数的位置为p,计算出区间[1,p-1]大于a[p]的数个数和区间[p+1,n]小于a[p]的数个数,更新答案,然后从序列中删掉该数。复杂度O(nlogn\^2)

TLE(常数大)

\begin{lstlisting}
struct treap_node {
    int rnd,sz,ls,rs,sum;
} T[MAXN<<5];
int tot;
void init() { srand(19981111); tot=0; }
struct treap {
    int root;
    void init(){ root=0; }
    #define lt T[x].ls
    #define rt T[x].rs
    int newnode(int val){ T[++tot]={rand(),1,0,0,val}; return tot; }
    void up(int x){ if (x) T[x].sz=T[lt].sz+T[rt].sz+1; }
    void split(int x, int& l, int& r, int k) {
        if (!x) { l=r=0; return; }
        if (k<=T[lt].sz) r=x, split(lt,l,lt,k), up(x);
        else l=x, split(rt,rt,r,k-T[lt].sz-1), up(x);
    }
    void Merge(int& x, int l, int r) {
        if (!l||!r) x=l+r;
        else if (T[l].rnd<T[r].rnd) x=l, Merge(rt,rt,r), up(x);
        else x=r, Merge(lt,l,lt), up(x);
    }
    int LT(int x, int val) {
        if (!x) return 0;
        if (val<=T[x].sum) return LT(lt,val);
        return T[lt].sz+1+LT(rt,val);
    }
    int GT(int x, int val) {
        if (!x) return 0;
        if (val>=T[x].sum) return GT(rt,val);
        return T[rt].sz+1+GT(lt,val);
    }
    void Insert(int val) {
        int x,y;
        split(root,x,y,LT(root,val));
        Merge(x,x,newnode(val)); Merge(root,x,y);
    }
    void del(int val) {
        int x,y,z,rnk=LT(root,val);
        split(root,x,y,rnk+1); split(x,x,z,rnk);
        Merge(root,x,y);
    }
    int LT(int val) { return LT(root,val); }
    int GT(int val) { return GT(root,val); }
    void show(int x) {
        if (x) {
            show(lt);
            printf(" %d", T[x].sum);
            show(rt);
        }
    }
    void show(){ show(root); }
} tree[MAXN<<2];
int n,a[MAXN];
#define ls x<<1
#define rs x<<1|1
#define mid (l+r)/2
void build(int x, int l, int r) {
    tree[x].init();
    FOR(i,l,r) tree[x].Insert(a[i]);
    if (l<r) {
        build(ls,l,mid);
        build(rs,mid+1,r);
    }
}
void del(int x, int l, int r, int p) {
    tree[x].del(a[p]);
    if (l<r) {
        if (p<=mid) del(ls,l,mid,p);
        else del(rs,mid+1,r,p);
    }
}
//[p+1,n] a[i] s.t. a[i]<a[p]
int LT(int x, int l, int r, int p) {
    if (p<l) return tree[x].LT(a[p]);
    int ans=LT(rs,mid+1,r,p);
    if (p<mid) ans+=LT(ls,l,mid,p);
    return ans;
}
//[1,p-1] a[i] s.t. a[i]>a[p]
int GT(int x, int l, int r, int p) {
    if (r<p) return tree[x].GT(a[p]);
    int ans=GT(ls,l,mid,p);
    if (mid+2<=p) ans+=GT(rs,mid+1,r,p);
    return ans;
}
int b[MAXN];
LL ans;
void merge_sort(int* A, int l, int r) {
    static int T[MAXN];
    if (r-l>1) {
        int m=(l+r)>>1;
        merge_sort(A,l,m);
        merge_sort(A,m,r);
        int i=l, j=m, k=l;
        while (i<m && j<r) {
            if (A[i]<=A[j]) T[k++] = A[i++];
            else {
                T[k++] = A[j++];
                ans+=m-i;
            }
        }
        while (i<m) T[k++] = A[i++];
        while (j<r) T[k++] = A[j++];
        memcpy(A+l,T+l,sizeof(int)*(r-l));
    }
}
void show(int x, int l, int r) {
    printf("[%d,%d]:", l,r); tree[x].show(); puts("");
    if (l<r) {
        show(ls,l,mid);
        show(rs,mid+1,r);
    }
}

int main()
{

    int n,q; scanf("%d%d", &n,&q);
    FOR(i,1,n) scanf("%d", &a[i]), b[i]=a[i];
    ans=0;
    merge_sort(b,1,n+1);
    map<int,vector<int> > mp;
    FOR(i,1,n) {
        mp[a[i]].pb(i);
    }
    init();
    build(1,1,n);
    while (q--) {
        int val; scanf("%d", &val);
        printf("%lld\n", ans);
        for (auto& p:mp[val]) {
            if (p!=1) ans-=GT(1,1,n,p);
            if (p!=n) ans-=LT(1,1,n,p);
            del(1,1,n,p);
        }
    }

    return 0;
}
\end{lstlisting}

P3759 [TJOI2017]不勤劳的图书管理员

动态逆序对 树状数组套主席树 模板

给定一排书,每本书有它本应该属于的位置和页数,每次可以选择两本书交换,要求计算每次交换后的贡献,每两本逆序的书产生它们的页数和的贡献。

树状数组套主席树减少空间的一个重要优化:更新的时候只有当结点为空结点时才动态开点。

树状数组套主席树就是应该像树状数组套一层主席树,写法简洁了不少

\begin{lstlisting}
int n,q,a[MAXN],b[MAXN];
inline void addmod(int& a, int b){ a=((LL)a+b+MOD)%MOD; }
#define mid ((l+r)>>1)
int sz[MAXN<<8],sum[MAXN<<8],L[MAXN<<8],R[MAXN<<8],tot,rt[MAXN];
void init(){ tot=0; }
void update(int& x, int l, int r, int p, int val, int sgn) {
    if (!x) x=++tot; //!!!
    sz[x]+=sgn; addmod(sum[x],sgn*val);
    if (l<r) {
        if (p<=mid) update(L[x],l,mid,p,val,sgn);
        else update(R[x],mid+1,r,p,val,sgn);
    }
}
void LT(int x, int l, int r, int k, int& ans, int& cnt) {
    while (l<k) {
        if (mid<k) {
            addmod(ans,sum[L[x]]);
            cnt+=sz[L[x]];
            x=R[x];
            l=mid+1;
        } else {
            x=L[x];
            r=mid;
        }
    }
}
void GT(int x, int l, int r, int k, int& ans, int& cnt) {
    while (k<r) {
        if (k<=mid) {
            addmod(ans,sum[R[x]]);
            cnt+=sz[R[x]];
            x=L[x];
            r=mid;
        } else {
            x=R[x];
            l=mid+1;
        }
    }
}
inline int lowbit(int x){ return x&-x; }
void add(int p, int sgn) {
    for (int i=p; i<=n; i+=lowbit(i)) update(rt[i],1,n,a[p],b[p],sgn);
}
void sol(int& ans, int p, int sgn) {
    int res,cnt;
    res=cnt=0;
    for (int i=p-1; i; i-=lowbit(i)) GT(rt[i],1,n,a[p],res,cnt);
    addmod(ans,sgn*((res+(LL)cnt*b[p])%MOD));
    res=cnt=0;
    for (int i=p; i; i-=lowbit(i)) LT(rt[i],1,n,a[p],res,cnt);
    res*=-1; cnt*=-1;
    for (int i=n; i; i-=lowbit(i)) LT(rt[i],1,n,a[p],res,cnt);
    addmod(ans,sgn*((res+(LL)cnt*b[p])%MOD));
    add(p,sgn);
}

int main()
{

    while (scanf("%d%d", &n,&q)==2) {
        init();
        int ans=0;
        FOR(i,1,n) {
            scanf("%d%d", &a[i],&b[i]);
            sol(ans,i,1);
        }
        while (q--) {
            int l,r; scanf("%d%d", &l,&r);
            if (l>r) swap(l,r);
            if (l<r) {
                sol(ans,l,-1);
                sol(ans,r,-1);
                swap(a[l],a[r]);
                swap(b[l],b[r]);
                sol(ans,l,1);
                sol(ans,r,1);
            }
            printf("%d\n", ans);
        }
    }

    return 0;
}
\end{lstlisting}