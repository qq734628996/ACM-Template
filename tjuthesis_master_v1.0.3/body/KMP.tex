\section{KMP}

P3375 【模板】KMP字符串匹配

一些细节:next数组第m个元素也是有值的,方便匹配成功后继续next。此题无需优化,最简单写法即可,next数组打印从1到m而不是从0到m-1。

\begin{lstlisting}
char T[MAXN],P[MAXN];
int nxt[MAXN],n,m;
void getnext()
{
    int j = 0, k = nxt[0] = -1;
    while (j < m) {
        //if (k==-1 || P[j]==P[k]) j++,k++,nxt[j] = P[j]!=P[k] ? k : nxt[k];
        if (k==-1 || P[j]==P[k]) nxt[++j] = ++k;
        else k = nxt[k];
    }
}
void KMP()
{
    int i = 0, j = 0;
    while (i < n) {
        if (j==-1 || T[i]==P[j]) i++,j++;
        else j = nxt[j];
        if (j==m) printf("%d\n", i-j+1);
    }
}

int main()
{

    scanf("%s%s", T,P);
    n = strlen(T), m = strlen(P);
    getnext(); KMP();
    FOR(i,1,m-1) printf("%d ", nxt[i]); printf("%d\n", nxt[m]);

    return 0;
}
\end{lstlisting}

Period UVALive - 3026

训练指南第三章例题13

KMPnext数组、最小循环节

如果前i个字符组成一个周期串,则“错位”部分恰好是一个循环节:满足nxt[i]>0,i%(i-nxt[i])==0,周期T=i-nxt[i],循环次数k=i/(i-nxt[i])。

此处next数组定义为“最长相同前后缀”

\begin{lstlisting}
int n,nxt[MAXN],kase;
char a[MAXN];
void get_next(char* P)
{
    int j=0, k=nxt[0]=-1;
    while (j<n) {
        if (k==-1 || P[j]==P[k]) nxt[++j] = ++k;
        else k = nxt[k];
    }
}

int main()
{

    while (scanf("%d", &n)==1 && n) {
        scanf("%s", a);
        get_next(a);
        printf("Test case #%d\n", ++kase);
        FOR(i,2,n) if (nxt[i]>0 && i%(i-nxt[i])==0)
            printf("%d %d\n", i,i/(i-nxt[i]));
        puts("");
    }

    return 0;
}
\end{lstlisting}

D - Cyclic Nacklace HDU - 3746

最小循环节

i-nxt[i]的含义是前i个元素中最小循环节的长度。

\begin{lstlisting}
int n,nxt[MAXN],kase;
char a[MAXN];
void get_next(char* P)
{
    int j=0, k=nxt[0]=-1;
    while (j<n) {
        if (k==-1 || P[j]==P[k]) nxt[++j] = ++k;
        else k = nxt[k];
    }
}

int main()
{

    int T; scanf("%d", &T);
    while (T--) {
        scanf("%s", a);
        n=strlen(a), get_next(a);
        int cir = n-nxt[n];
        if (n%cir==0 && n/cir>1) puts("0");
        else printf("%d\n", cir-n%cir);
    }

    return 0;
}
\end{lstlisting}

扩展KMP模板

扩展KMP解决S所有后缀子串与T的最长公共前缀长度问题。

next数组含义:T[i,m-1]与T的最长相同前缀长度

extend数组含义:S[i,n-1]与T的最长相同前缀长度

\begin{lstlisting}
int n,m,nxt[MAXN],ext[MAXN];
void get_next(char* T)
{
    int a=0, p=0;
    nxt[0] = m;
    FOR(i,1,m-1) {
        if (i>=p || i+nxt[i-a]>=p) {
            if (i>=p) p=i;
            while (p<m && T[p]==T[p-i]) p++;
            nxt[i] = p-i;
            a = i;
        } else nxt[i] = nxt[i-a];
    }
}
void extKMP(char* S, char* T)
{
    int a=0, p=0;
    REP(i,n) {
        if (i>=p || i+nxt[i-a]>=p) {
            if (i>=p) p=i;
            while (p<n && p-i<m && S[p]==T[p-i]) p++;
            ext[i] = p-i;
            a = i;
        } else ext[i] = nxt[i-a];
    }
}
\end{lstlisting}