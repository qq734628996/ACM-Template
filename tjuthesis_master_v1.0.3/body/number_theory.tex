\section{数论}

\subsection{素数筛}

\begin{lstlisting}
int vis[MAXN],prime[MAXN],tot;
void getPrime(int n)
{
    FOR(i,2,n) {
        if (!vis[i]) prime[tot++]=i;
        REP(j,tot) {
            if (i*prime[j]>n) break;
            vis[i*prime[j]]=1;
            if (i%prime[j]==0) break;
        }
    }
}
\end{lstlisting}

\subsection{素性测试}

test\_time 为测试次数,建议设为不小于 8 的整数以保证正确率,但也不宜过大,否则会影响效率

\begin{lstlisting}
bool millerRabbin(int n) {
  if (n < 3) return n == 2;
  int a = n - 1, b = 0;
  while (a % 2 == 0) a /= 2, ++b;
  for (int i = 1, j; i <= test_time; ++i) {
    int x = rand() % (n - 2) + 2, v = quickPow(x, a, n);
    if (v == 1 || v == n - 1) continue;
    for (j = 0; j < b; ++j) {
      v = (long long)v * v % n;
      if (v == n - 1) break;
    }
    if (j >= b) return 0;
  }
  return 1;
}
\end{lstlisting}

\subsection{反素数}

如果某个正整数 $n$ 满足如下条件,则称为是反素数:任何小于 $n$ 的正数的约数个数都小于 $n$ 的约数个数

https://blog.csdn.net/ACdreamers/article/details/25049767

Number With The Given Amount Of Divisors CodeForces - 27E

求约数个数为n的最小整数

\begin{lstlisting}
typedef unsigned long long ull;
int p[16]={2,3,5,7,11,13,17,19,23,29,31,37,41,43,47,53};
ull ans;
int n;
void dfs(int dep, ull now, int num, int up)
{
    if (num>n || dep>=16) return;
    if (num==n && ans>now) {
        ans=now;
        return;
    }
    FOR(i,1,up) {
        if (now/p[dep]>ans) break;
        dfs(dep+1,now=now*p[dep],num*(i+1),i);
    }
}

int main()
{

    while (scanf("%d", &n)==1) {
        ans=~0ULL;
        dfs(0,1,1,60);
        printf("%llu\n", ans);
    }

    return 0;
}
\end{lstlisting}

More Divisors ZOJ - 2562

求n以内因子数最多的数

\begin{lstlisting}
typedef unsigned long long ull;
int p[16]={2,3,5,7,11,13,17,19,23,29,31,37,41,43,47,53};
ull n,ans,ans_num;
void dfs(int dep, ull now, int num, int up)
{
    if (dep>=16 || now>n) return;
    if (num>ans_num || (num==ans_num && ans>now)) {
        ans=now;
        ans_num=num;
    }
    FOR(i,1,up) {
        if (now*p[dep]>n) break;
        dfs(dep+1,now*=p[dep],num*(i+1),i);
    }
}

int main()
{

    while (scanf("%llu", &n)==1) {
        ans_num=0;
        dfs(0,1,1,60);
        printf("%llu\n", ans);
    }

    return 0;
}
\end{lstlisting}

小明系列故事——未知剩余系 HDU - 4542

给出一个数K和两个操作

如果操作是0,就求出一个最小的正整数X,满足X的约数个数为K。

如果操作是1,就求出一个最小的X,满足X的约数个数为X-K。

https://blog.csdn.net/qq\_37025443/article/details/75007451

上面的模板有错,以这个为准

\begin{lstlisting}
const LL inf=(1LL<<62)+1;
int p[16]={2,3,5,7,11,13,17,19,23,29,31,37,41,43,47,53};
LL ans;
int op,n,d[MAXN];
void init()
{
    int m=MAXN-5;
    FOR(i,1,m) d[i]=i;
    FOR(i,1,m) {
        for (int j=i; j<=m; j+=i) d[j]--;
        if (!d[d[i]]) d[d[i]]=i;
        d[i]=0;
    }
}
void dfs(int dep, LL now, int num, int up)
{
    if (num>n) return;
    if (num==n && ans>now) { ans=now; return; }
    FOR(i,1,up) {
        if (ans/p[dep]<now || num*(i+1)>n) break;
        now*=p[dep];
        if (n%(num*(i+1))==0) dfs(dep+1,now,num*(i+1),i);
    }
}

int main()
{

    init();
    int T; scanf("%d", &T);
    FOR(kase,1,T) {
        scanf("%d%d", &op,&n);
        if (op) ans=d[n];
        else { ans=inf; dfs(0,1,1,62); }
        printf("Case %d: ", kase);
        if (!ans) puts("Illegal");
        else if (ans>=inf) puts("INF");
        else printf("%lld\n", ans);
    }

    return 0;
}
\end{lstlisting}

\subsection{欧几里得算法}

\begin{lstlisting}
//ax+by=d=gcd(a,b)
LL gcd(LL a, LL b) { return b==0?a:gcd(b,a%b); }
LL lcm(LL a, LL b) { return a/gcd(a,b)*b; }
void ex_gcd(LL a, LL b, LL& d, LL& x, LL& y)
{
    if (!b) x=1, y=0, d=a;
    else ex_gcd(b,a%b,d,y,x), y-=x*(a/b);
}
\end{lstlisting}

\subsection{快速幂取模}

\begin{lstlisting}
LL powMod(LL x, LL e)
{
    if (!e) return 1;
    return e&1 ? powMod(x,e-1)*x%MOD : powMod(x*x%MOD,e>>1);
}
LL powMod(LL x, LL e, LL p)
{
    if (!e) return 1;
    return e&1 ? powMod(x,e-1,p)*x%p : powMod(x*x%p,e>>1,p);
}
\end{lstlisting}

\subsection{欧拉函数}

\begin{lstlisting}
int vis[MAXN],prime[MAXN],phi[MAXN],tot;
void getphi(int n)
{
    phi[1]=1;
    FOR(i,2,n) {
        if (!vis[i]) prime[tot++]=i, phi[i]=i-1;
        REP(j,tot) {
            if (i*prime[j]>n) break;
            vis[i*prime[j]]=1;
            if (i%prime[j]==0) {
                phi[i*prime[j]]=phi[i]*prime[j];
                break;
            } else phi[i*prime[j]]=phi[i]*(prime[j]-1);
        }
    }
}
\end{lstlisting}

\subsection{逆元}

\begin{lstlisting}
void ex_gcd(LL a, LL b, LL& d, LL& x, LL& y)
{
    if (!b) x=1, y=0, d=a;
    else ex_gcd(b,a%b,d,y,x), y-=x*(a/b);
}
LL inv(LL a, LL p)
{
    LL d,x,y;
    ex_gcd(a,p,d,x,y);
    return d==1?(x+p)%p:-1;
}
LL inv(LL a) { return powMod(a,MOD-2); }
\end{lstlisting}

线性递推求乘法逆元

\begin{lstlisting}
int inv[MAXN];
void getinv(int n, LL p)
{
    inv[1]=1;
    FOR(i,2,n) inv[i]=(p-p/i)*inv[p%i]%p;
}
\end{lstlisting}

阶乘逆元

\begin{lstlisting}
LL powMod(LL x, LL e)
{
    if (!e) return 1;
    return e&1 ? powMod(x,e-1)*x%MOD : powMod(x*x%MOD,e>>1);
}
LL fac[MAXN],inv[MAXN];
void getinv(int n)
{
    fac[0]=1;
    FOR(i,1,n) fac[i]=fac[i-1]*i%MOD;
    inv[0]=1;
    inv[n]=powMod(fac[n],MOD-2);
    ROF(i,1,n-1) inv[i]=inv[i+1]*(i+1)%MOD;
}
\end{lstlisting}

\subsection{模方程}

模板

中国剩余定理,孙子定理

//n个方程:x=a[i](mod m[i]) (0<=i<n)

\begin{lstlisting}
void ex_gcd(LL a, LL b, LL& d, LL& x, LL& y)
{
    if (!b) x=1, y=0, d=a;
    else ex_gcd(b,a%b,d,y,x), y-=x*(a/b);
}
LL china(int n, int* a, int* m)
{
    LL M=1,d,y,x=0;
    REP(i,n) M*=m[i];
    REP(i,n) {
        LL w=M/m[i];
        ex_gcd(m[i],w,d,d,y);
        x=(x+y*w*a[i])%M;
    }
    return (x+M)%M;
}
\end{lstlisting}

离散对数

又叫BSGS算法(Shank's Baby-Step-Giant-Step Algorithm,Shank的大步小步算法)

//求解模方程a\^x=b(mod n)。n为素数,无解返回-1

\begin{lstlisting}
void ex_gcd(LL a, LL b, LL& d, LL& x, LL& y)
{
    if (!b) x=1, y=0, d=a;
    else ex_gcd(b,a%b,d,y,x), y-=x*(a/b);
}
LL inv(LL a, LL p)
{
    LL d,x,y;
    ex_gcd(a,p,d,x,y);
    return d==-1?(x+p)%p:-1;
}
LL powMod(LL x, LL e, LL p)
{
    if (!e) return 1;
    return e&1 ? powMod(x,e-1,p)*x%p : powMod(x*x%p,e>>1,p);
}
//a^x=b(mod n), n is prime
int log_mod(int a, int b, int n)
{
    int m=sqrt(n+0.5);
    int v=inv(powMod(a,m,n),n);
    map<int,int> x;
    x[1]=0;
    int e=1;
    for (int i=1; i<m; i++) {
        e=e*a%n;
        if (!x.count(e)) x[e]=i;
    }
    REP(i,m) {
        if (x.count(b)) return i*m+x[b];
        b=b*v%n;
    }
    return -1;
}
\end{lstlisting}