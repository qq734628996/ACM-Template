\section{最大流dinic}

洛谷 P3376

最大流模板 dinic

\begin{lstlisting}
struct dinic {
    const static LL inf=0x3f3f3f3f3f3f3f3f;
    struct edge {
        int from,to;
        LL cap,flow;
    };
    int n,m,s,t;
    vector<edge> edges;
    vector<int> G[MAXN];
    bool vis[MAXN];
    int d[MAXN],cur[MAXN];
    void init(int n) {
        this->n=n;
        edges.clear();
        FOR(i,0,n) G[i].clear();
    }
    void add(int from, int to, LL cap) {
        edges.pb({from,to,cap,0});
        edges.pb({to,from,0,0});
        m=SZ(edges);
        G[from].pb(m-2);
        G[to].pb(m-1);
    }
    bool bfs() {
        memset(vis,0,sizeof(vis));
        queue<int> Q;
        Q.push(s); d[s]=0; vis[s]=1;
        while (!Q.empty()) {
            int u=Q.front(); Q.pop();
            REP(i,SZ(G[u])) {
                edge& e=edges[G[u][i]];
                if (!vis[e.to] && e.cap>e.flow) {
                    vis[e.to]=1;
                    d[e.to]=d[u]+1;
                    Q.push(e.to);
                }
            }
        }
        return vis[t];
    }
    LL dfs(int u, LL a) {
        if (u==t || !a) return a;
        LL flow=0, f;
        for (int& i=cur[u]; i<SZ(G[u]); i++) {
            edge& e=edges[G[u][i]];
            if (d[e.to]==d[u]+1 && (f=dfs(e.to,min(a,e.cap-e.flow)))>0) {
                e.flow+=f;
                edges[G[u][i]^1].flow-=f;
                flow+=f;
                a-=f;
                if (!a) break;
            }
        }
        return flow;
    }
    LL maxflow(int s, int t) {
        this->s=s, this->t=t;
        LL flow=0;
        while (bfs()) {
            memset(cur,0,sizeof(cur));
            flow+=dfs(s,inf);
        }
        return flow;
    }
} solve;

int main()
{

    int n,m,s,t;
    while (scanf("%d%d%d%d", &n,&m,&s,&t)==4) {
        solve.init(n+1);
        REP(i,m) {
            int x,y,z; scanf("%d%d%d", &x,&y,&z);
            solve.add(x,y,z);
        }
        printf("%lld\n", solve.maxflow(s,t));
    }

    return 0;
}
\end{lstlisting}

Reactor Cooling ZOJ - 2314

无源无汇有容量下界网络的可行流

新增源点s、汇点t,t到s添加一条无穷大的边,对于每条弧u->v,[b,c],拆成3条:u->t (b), u->v (c-b), s->v (b),有可行流当且仅当所有附加弧满载

注意此题并没有考虑重边的情况

\begin{lstlisting}
int main()
{

    int T; scanf("%d", &T);
    while (T--) {
        int n,m; scanf("%d%d", &n,&m);
        solve.init(n+5);
        int s=0,t=n+1;
        solve.add(t,s,dinic::inf);
        map<pii,int> bbak;
        REP(i,m) {
            int u,v,b,c; scanf("%d%d%d%d", &u,&v,&b,&c);
            bbak[{u,v}]=b;
            solve.add(u,t,b);
            solve.add(u,v,c-b);
            solve.add(s,v,b);
        }
        solve.maxflow(s,t);
        bool ok=true;
        for (auto& e:solve.edges) {
            if ((e.from==s || e.to==t) && e.cap>e.flow) ok=false;
        }
        if (!ok) puts("NO");
        else {
            puts("YES");
            for (auto& e:solve.edges) {
                if (bbak.count({e.from,e.to})) {
                    printf("%lld\n", e.flow+bbak[{e.from,e.to}]);
                }
            }
        }
        if (T) puts("");
    }

    return 0;
}
\end{lstlisting}